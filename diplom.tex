\documentclass[%
master,    % тип документа
natbib,      % использовать пакет natbib для "сжатия" цитирований
subf,        % использовать пакет subcaption для вложенной нумерации рисунков
href,        % использовать пакет hyperref для создания гиперссылок
colorlinks,  % цветные гиперссылки
%fixint,     % включить прямые знаки интегралов
]{disser}

\usepackage[
a4paper, mag=1000,
left=2.5cm, right=1cm, top=2cm, bottom=2cm, headsep=0.7cm, footskip=1cm
]{geometry}

\usepackage[intlimits]{amsmath}
\usepackage{amssymb,amsfonts}

\usepackage[T2A]{fontenc}
\usepackage[utf8]{inputenc}
\usepackage[english,russian]{babel}
\ifpdf\usepackage{epstopdf}\fi
\usepackage[autostyle]{csquotes}

% Шрифт Times в тексте как основной
%\usepackage{tempora}
\usepackage{setspace}
% альтернативный пакет из дистрибутива TeX Live
%\usepackage{cyrtimes}

% Шрифт Times в формулах как основной
%\usepackage[varg,cmbraces,cmintegrals]{newtxmath}
% альтернативный пакет
%\usepackage[subscriptcorrection,nofontinfo]{mtpro2}

% Плавающие рисунки "в оборку".
\usepackage{wrapfig}

% Номера страниц снизу и по центру
%\pagestyle{footcenter}
%\chapterpagestyle{footcenter}

% Точка с запятой в качестве разделителя между номерами цитирований
%\setcitestyle{semicolon}

% Использовать полужирное начертание для векторов
\let\vec=\mathbf
%______________________________-
\usepackage{lipsum}

%\usepackage{titlesec}
%\usepackage{spacing}
%\titleformat{\section}[block]{\color{blue}\Large\bfseries\filcenter}{}{1em}{}
% Номера страниц снизу и по центру
\pagestyle{footcenter}
\chapterpagestyle{footcenter}

% Точка с запятой в качестве разделителя между номерами цитирований
%\setcitestyle{semicolon}



% Переопределение стандартных заголовков
%\def\contentsname{Содержание}
%\def\conclusionname{Выводы}
%\def\bibname{Литература}

\usepackage{geometry} % пакет для установки полей
\geometry{top=1.5cm} % отступ сверху
\geometry{bottom=2cm} % отступ снизу
\geometry{left=3cm} % отступ справа
\geometry{right=1.5cm} % отступ слева
\newcommand{\sectionbreak}{\clearpage}
\newcommand*{\No}{\textnumero}
\renewcommand{\Re}{\mathrm{Re}}
\renewcommand{\Im}{\mathrm{Im}}

\newcommand{\const}{\mathrm{const}}
\newcommand{\arccosh}{\mathrm{arccosh}}

\newcommand{\vF}{\mathbf{F}}
\newcommand{\ve}{\mathbf{e}}
\newcommand{\vk}{\mathbf{k}}
\newcommand{\vq}{\mathbf{q}}
\newcommand{\vp}{\mathbf{p}}
\newcommand{\va}{\mathbf{a}}
\newcommand{\vP}{\mathbf{P}}
\newcommand{\vK}{\mathbf{K}}
\newcommand{\vQ}{\mathbf{Q}}
\newcommand{\vA}{\mathbf{A}}
\newcommand{\vr}{\mathbf{r}}
\newcommand{\vR}{\mathbf{R}}

\newcommand{\vRR}{\boldsymbol{\mathcal{R}}}
\newcommand{\veps}{\boldsymbol{\varepsilon}}

\newcommand{\cA}{\mathcal{A}}
\newcommand{\cR}{\mathcal{R}}
\newcommand{\cM}{\mathcal{M}}
\newcommand{\cE}{\mathcal{E}}
\newcommand{\cJ}{\mathcal{J}}
\newcommand{\cT}{\mathcal{T}}
\newcommand{\cD}{\mathcal{D}}


%______________________________-
% Включать подсекции в оглавление
\setcounter{tocdepth}{2}

\graphicspath{{fig/}}

\begin{document}
\title{Моделирование нуклеосинтеза в звездах}
\maketitle
\section*{\centering Введение}
\addcontentsline{toc}{section}{Введение}
Вопрос о том, из чего состоит материальный мир стоит перед учеными с самого зарождения науки Левкипп (около 430 г. до н.э.) и Демокрит (около 420 г. до н.э.) первыми предложили атомную теорию, в которой вся материя состоит из неделимых частиц. Позже ученые добились успехов в экспериментах с процессами возникновения различных веществ. Алхимики, например, задавались вопросами преобразования обычных металлов (свинца, например) в благородные (такие как золото). Попытки их были тщетны, и теоретическая основа этих преобразований не получила никакого развития. И только в конце XX века ядерные физики добились успеха превращения висмута в золото (лишь в небольших количествах и с коммерческими расходами)
В связи с развитием ядерной физики были также построено огромное количество различных математических моделей, объясняющих возникновение тяжелых элементов, а именно тяжелее железа, из более легких.

В данной работе, как видно из названия, я буду моделировать такие реакции с использование открытой библиотеки реакций ReacLib, в основе которой лежит построение сечения зависимостью от температуры по 7 параметрам. 

Основной целью работы является построение сечений для столкновительного $\beta$-распада при столкновении элементов с протоном, а также оценка влияния этих реакций на полученную распространенность элементов в результате всех процессов за промежуток времени.

Сам процесс моделирования будет выполняться с помощью открытой библиотеки SkyNet, написанную Jonas Lippuner с дополнение ее своим набором реакций.
\section{ Столкновительный $\beta$-распад}


\subsection{Распространенность в солнечной системе}


\subsection{Нуклеосинтез Большого Взрыва}
\subsection{Ядерное горение в тяжелых звездах}
\subsection{Пик железа}
\subsection{Нуклеосинтез за пиком железа}
\subsection{Возможные места r-процесса}
\subsection{Коллапс ядра сверхновых}
\subsection{Химическая эволюция галактики}
\subsection{Эволюция само-нагрева}
\subsection{2.3.3 Критерий сходимости и временной шаг}

\end{document} 