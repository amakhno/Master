\chapter{Проблема происхождения обойденных изотопов.}

Прямые эксперименты по наблюдению  СБР ядер  сложны, так как
ввиду малости сечения явление будет замаскировано различными фоновыми
помехами. Однако, существует возможность получения косвенных свидетельств
в пользу реальности СБР, и связана она с некоторыми нерешенными проблемами
ядерной астрофизики.

  Ядерная астрофизика была и остается одной из наиболее активных областей
современных исследований. Достижения современной экспериментальной и теоретической
ядерной физики, рассматриваемые в связи с астрономическими объектами,
помогают решать такие важные проблемы, как происхождение химических элементов,
самые поздние стадии эволюции звезд, предшествующие вспышкам сверхновых и новых,
космохронология и другие.
В настоящее время одной из нерешенных проблем ядерной астрофизики является объяснение
наблюдаемых распространенностей
обойденных изотопов, кратко обсужденная во Введении. Рассмотрим теперь этот
вопрос детальнее.


\section{Астрофизические предпосылки \\ для модели образования
обойденных изотопов на основе
процесса столкновительного $\beta$-распада.}

Теория ядра диктует условия, при которых может встречаться процесс СБР стабильных
ядер, а астрофизика определяет подходящее для него место. Выше было показано, что
для осуществления процесса столкновительного $\beta$-распада требуется выполнение
следующего условия: кинетическая энергия столкновения должна превышать порог $\Delta$,
определяемый разностью значений энергии основных состояний дочернего и материнского
ядер, и ему не препятствует кулоновский барьер
(для увеличения интенсивности процесса желательно, кроме того, выполнение
квантовых правил отбора для разрешенных $\beta$-переходов).
Поэтому из-за максвелловского
распределения скоростей в квазиравновесной среде процесс возможен всегда
(за счет "хвостов" максвелловского распределения столкновительной энергии), и речь
может идти только об оценке его интенсивности. Таким образом, для реализации
этого процесса не требуется катастрофическая стадия, хотя интенсивные столкновения
частиц с ростом температуры и при взрыве сверхновой звезды его не устраняют,
а даже усиливают. Поэтому, чтобы подобрать наиболее благоприятные условия протекания
процесса столкновительного $\beta$-распада необходимо хотя бы кратко рассмотреть
эволюцию  звезды с позиций теории ядерных реакций.


Анализ особенностей кривой естественной распространенности атомных ядер в виде зависимости их содержаний от атомной массы
(см. рис. \ref{RASPR}), наводят на мысль, что наиболее вероятным источником  для большинства изотопов являются
определенные последовательности дискретных процессов, протекающих в недрах звезд, т.е. вполне конкретные группы ядерных
реакций. С учетом современных данных эти процессы, перечисленные Бербиджами, Фаулером и Хойлом в 1957 г. \cite{burbidge},
называются теперь следующим образом: водородное, гелиевое, углеродное, неоновое, кислородное и кремниевое горение, а
также $s$-, $r$-, $p$- и $l$-процессы \cite{leng, kosmos, iader, 97}.

\begin{figure}
\vspace{18 true cm}
\caption{{Кривая распространенности нуклидов в первичной
солнечной туманности по отношению к кремнию ($10^6$).}}
\label{RASPR}
\end{figure}


При температуре $\sim 10^6 K$ начинаются первые ядерные реакции -- выгорают
дейтерий, литий, бор. Первичное количество этих элементов настолько мало, что их
выгорание практически не задерживает сжатия. Сжатие прекращается, когда температура в центре
звезды достигает $\sim 10^7 K$ и загорается водород.

При горении водорода происходит слияние четырех протонов с образованием ядра
$ ^4 He$; этот процесс осуществляется либо  в про\-тон-про\-тон\-ной цепочке реакций,
либо в циклах ядерных реакций с участием углерода, азота и кислорода (возможно,
также и более тяжелых ядер) в качестве катализаторов. Подавляющее  количество
энергии, требуемой для поддержания огромной светимости звезд в течение большей
части их жизни, обеспечивается процессом превращения водорода в гелий, который
протекает сначала в центральных ядрах звезд, а затем в относительно тонких
сферических слоях, окружающих гелиевые ядра. Для такого процесса характерна
температура $T>10^7 K$ и время $t\approx 10^{10}$ лет.

Когда вследствии сжатия под действием собственной гравитации гелиевое ядро
проэволюционировавшей звезды становится достаточно горячим и плотным,
в нем начинается горение гелия. При этом происходит слияние ядер гелия и
образуются углерод и кислород при температуре $T>10^8$. Характерное время процесса
$t \approx 10^7 $ лет.

Затем в зависимости от массы звезда либо полностью разрушится
при взрывообразном загорании углерода в условиях сильного вырождения звездного
вещества, либо начнется горение углерода в гидростатически равновесном режиме,
когда сила тяжести в точности уравновешивается тепловым давлением.
Характерное время процесса $\approx 10^5$ лет, если нуклеосинтез
не носит взрывного характера, в противном случае ---
несколько секунд.

Если звезда "переживет" горение углерода, то затем в ее эволюции может быть короткая стадия,
в которой образовавшийся при горении углерода неон подвергается фотодиссоциации на компоненты
$^{16} O +^4 He$. Освобождающиеся таким путем $\alpha$-частицы могут соединяться
с недиссоциировавшими ядрами неона и образовывать $^{24}Mg$.
Если до этого момента сохранится какое-либо количество углерода, то он затем полностью
перерабатывается в кислород. В результате такого процесса, называемого теперь горением
неона, у звезды появляется центральное ядро, состоящее в основном из $^{16}O$
и $^{24}Mg$.

В ходе дальнейшего сжатия температура звезды повышается до $T>10^9 K$, т.е.
до тех пор, пока не начнется слияние ядер кислорода с выделением $\alpha$-частиц,
протонов и нейтронов, которые участвуют в образовании целого ряда элементов
примерно от $^{28}Si$ до $^{45}Sc$.

Эти последовательные эволюционные стадии звезды сменяют друг друга через все
уменьшающиеся промежутки времени, и стадия горения кислорода, возможно, сливается
со следующей эволюционной стадией горения кремния, которое протекает при температурах
несколько миллиардов кельвинов за очень короткое время (поряка секунды).
В результате фотодиссоциации от атомных ядер отделяются $\alpha$-частицы,
протоны и нейтроны и становится возможным образование целого ряда элементов
вплоть до элементов, лежащих в районе железного пика на кривой распространенности.

В районе железа энергия связи на нуклон достигает максимума, и поэтому
на фоторасщепление ядер элементов, лежащих за железным пиком, требуется затратить
больше энергии, чем выделится при добавлении нуклонов к ядрам. По этой причине,
а также вследствие значительного увеличения кулоновских барьеров с возрастанием
$Z$, нуклеосинтез с участием заряженных частиц должен в основном закончиться
на стадии эволюции звезды, которая соответствует образованию элементов вблизи
железного пика. Элементы тяжелее железа преимущественно должны быть продуктами
нуклеосинтеза при захвате нейтронов.

В отсутствии дальнейших источников ядерной энергии, которые необходимы для создания
теплового давления, препятствующего сжатию ядра звезды под действием гравитации, ядро
должно неизбежно коллапсировать. Коллапс ядра звезды, по-видимому, сопровождается
сильным гидродинамическим "отскоком", в результате которого во внешние области звезды
распространяется ударная волна, поджигающая несгоревшее ядерное топливо и выбрасывающая
большую часть массы звезды в межзвездное пространство в виде гигантского взрыва-вспышки
сверхновой звезды.

Указанные выше $s$-, $r$-, $p$- и $l$-процессы -- это соответственно захват нейтронов
с большим (slow -- медленный) и коротким (rapid -- быстрый) характерным временем, образование редких
богатых протонами (proton) тяжелых ядер и, наконец,
образование очень легких (light) ядер $Li$, $Be$ и $B$, которые легче разрушаются,
чем образуются в термоядерных реакциях. Вероятнее всего, эти легкие ядра
образуются в реакциях скалывания более распространенных нуклидов, таких, как углерод,
азот и кислород (реакция заключается в частичном разрушении сложного ядра при столкновении
с легким ядром, таким, как $H$ и $He$), однако место протекания этого
процесса пока еще не установлено достаточно надежно.

Нас интересует роль столкновительного $\beta$-распада в образовании тяжелых ядер.
Наиболее распространенные изотопы элементов тяжелее железа сформировались,
очевидно, в недрах массивных звезд  в результате последовательных реакций захвата нейтронов.
Ряд характерных особенностей хода кривой распространенности этих тяжелых ядер указывает на то,
что процесс их построения должен протекать достаточно эффективно как на сравнительно продолжительной
равновесной стадии эволюции звезд в условиях малых интенсивностей потока нейтронов ($s$-процесс),
так и в момент взрыва звезды при высокой интенсивности потока нейтронов ($r$-процесс).

%Согласно фундаментальной теории происхождения химических элементов тяжелые ядра
%(тяжелее железа) образуются в трех принципиально разных процессах:
%$r$-процессе, $s$-процессе и $p$-процессе. Первые два вносят основной вклад в образование тяжелых
%элементов, вклад третьего процесса невелик.

$S$-процесс представляет собой медленный захват нейтронов, при котором образовавшиеся неустойчивые ядра распадаются
раньше, чем успеет присоединиться следующий нейтрон. Этот процесс характеризуется умеренными плотностями потока нейтронов
($10^{15} - 10^{16}  нейтрон\cdot см^{-2} с^{-1}$) и протекает в недрах звезд при их эволюции. Что касается эволюционного
статуса звезд, ответственных за образование ядер $s$-процесса, наблюдаемых в Солнечной системе, то это почти наверняка
красные гиганты \cite{iader}.

Также общепризнано, что многие ядра тяжелее железа, включая все ядра за $^{209}Bi$,
синтезируются в $r$-процессе. Скорости захвата нейтронов должны быть больше скоростей $\beta$-распада.
В таком случае ядро в условиях высокой концентрации свободных нейтронов и
высокой температуры захватывает нейтроны путем реакций $(n, \gamma)$, которые протекают
быстрее по сравнению с $\beta$-распадами. Захваты нейтронов продолжаются до тех пор,
пока скорость реакции $(n, \gamma)$ не уравновесится со скоростью реакции $(\gamma, n)$.
После этого ядро "ждет", пока произойдет $\beta$-распад, что позволит ему снова
захватывать нейтроны. В результате трек, вдоль которого идет $r$-процесс, обычно
отстоит от полосы $\beta$-стабильности на 10 нейтронов в направлении нейтроноизбыточных изотопов.
Процесс прекращается, либо когда концентрация нейтронов и температура падают
настолько, что прекращаются реакции $(n, \gamma)$ и $(\gamma, n)$, либо когда
в результате деления наступает циклический процесс, при котором синтез более тяжелых ядер не возможен.

На кривой распространенности ядер (рис.\ref{RASPR}), синтезирумых в $r$-процессе, имеются пики
около атомных масс $80, 130$ и $195$, которые коррелируют с магическими числами нейтронов соответственно
$50, 82$ и $126$. Пики в содержаниях ядер, связанные с теми же магическими числами
нейтронов, но расположенные при несколько больших атомных массах, наблюдаются также у ядер,
синтезируемых в $s$-процессе. Это является отражением того факта, что трек
$r$-процесса проходит в нейтроноизбыточной области далеко от полосы стабильности,
в то время как трек $s$-процесса идет по полосе стабильности. Треки  $s$- и $r$-процессов синтеза ядер
показаны на рис. \ref{TR}. Считается, что двойные максимумы вблизи магических чисел служат убедительным
доказательством существования в природе двух указанных процессов нейтронного захвата:
пики на кривой распространенности, связанные с $s$- и $r$-процессами, примерно одинаковы,
хотя сами процессы, по-видимому, абсолютно различны.

\begin{figure}
\vspace{18 true cm}
\caption{{Трек, вдоль которого идет захват нейтронов в $s$-  и $r$-процессах.}}
\label{TR}
\end{figure}


До сих пор, однако, не решен вопрос, при каком именно астрофизическом явлении
происходит образование ядер $r$-процесса. В качестве возможных кандидатов рассматривались области,
непосредственно окружающие нейтронизованные ядра взрывающихся сверхновых \cite{burbidge},
ударные волны в сверхновых \cite{sch12, sch24}, новые \cite{sch12, sch39},
гидродинамическая неустойчивость вращающегося замагниченного ядра звезды \cite{sch49, sch56, sch69},
столкновение нейтронных звезд с черными дырами  \cite{sch48}
и прохождение ударной волны по гелиевой и углеродной
зонам с сверхновых \cite{sch34, sch52, sch55, sch86}.
В настоящее время слишком мало известно об общей картине начальных
условий, необходимых для образования ядер $r$-процесса, чтобы исключить какую-либо из этих возможностей.
Но с другой стороны, если бы $r$-процесс происходил при различных физических условиях, меняющихся в
широком диапазоне и в каждом отдельном случае дающих свою картину распространенностей, то следовало бы
ожидать, что наблюдаемые пики распространенностей скорее будут широкими и не очень явно выраженными.
Узость этих пиков свидетельствует о том, что сам трек $r$-процесса также не слишком широк, а отсюда
следует, что либо диапазон условий, при которых образовалась основная масса $r$-ядер, должен быть ограниченным,
либо есть какая-то причина, направляющая $r$-процесс всегда по одному и тому же треку.
На эти вопросы астрофизикам еще предстоит ответить. Мы же пока будем ориентироваться на следующие
характерные для $r$-процесса значения: высокие плотности потока нейтронов
$(10^{27} - 10^{40} нейтрон \cdot см^{-2} с^{-1})$, температура $T>10^{10} K$
и время $1-100 \; секунд$.

Итак, согласно фундаментальной теории происхождения химических элементов
более тяжелые ядра возникают
из легких путем захвата нейтронов (s- и r-процессы) и последующего
$\beta^-$ - распада продуктов такой реакции. Однако, такая схема оказывается
бессильной при попытке объяснить происхождение обойденных ядер. Проблема
их синтеза по указанному механизму, как уже отмечалось, состоит в том, что цепочка последовательных
$\beta^-$-превращений ядер, образовавшихся при захвате нейтронов, заканчивается
стабильным изотопом $(A,Z)$ из главной последовательности.
Переходу к обойденному ядру $(A,Z+2)$, которое также стабильно, мешает энергетический
порог $\Delta$ предыдущего распада $(A,Z)\rightarrow (A,Z+1)$.
В качестве примера приведем несколько цепочек из трех изобар, в которых
первый элемент - $\beta$-стабильное ядро из главной последовательности,
второй от первого отделен порогом $\Delta$, а последний - обойденный изотоп:
$^{74}Ge\,-\,{}^{74}As\,-\,{}^{74}Se$, ${}^{106}Pd\,-\,{}^{106}Ag\,-\,{}^{106}Cd$,
${}^{164}Dy\,-\,{}^{164}Ho\,-\,{}^{164}Fr$ (всего наблюдается более 30 аналогичных
триад в диапазоне зарядовых чисел $34\leq Z\leq 80$).
На рис. \ref{FR} представлен фрагмент схемы последовательных \be-превращений
нуклидов с $A=80$. Обойденный изобар $^{80}Kr$.

\begin{figure}
\vspace{18 true cm}
\caption{{Фрагмент схемы последовательных \be-превращений
нуклидов с $A=80$. Обойденный изобар $^{80}Kr$.}}
\label{FR}
\end{figure}

Физический механизм СБР ядер в кулоновском поле можно использовать в качестве
средства преодоления энергетического порога $\Delta$ и реализации
$\bm$ - перехода $(A,Z)\rightarrow (A,Z+1)$, запрещенность которого
как раз и прерывала цепочку на элементе $(A,Z)$, не позволяя осуществиться
естественному $\bm$ - распаду $(A,Z+1)\rightarrow (A,Z+2)$. Как уже упоминалось
ранее, наличие кулоновского барьера для ядро-ядерных реакций процессу СБР стабильного
ядра не препятствует.

В этом случае необходимо, чтобы в звездной среде в $s$- или $r$-процессе
уже образовались тяжелые стабильные ядра из главной последовательности --
прародители обойденных ядер. Это значит, что образование обойденных изотопов
на основе столкновительного $\beta$-распада возможно только, когда
происходят $s$- и $r$-процессы или после них.

\section{Столкновительная модель образования  обойденных  ядер
в звездном веществе на квазиравновесной стадии эволюции.}


Сформулируем следующую столкновительную модель синтеза обойденных  изотопов
в звездном веществе. Пусть звездная среда находится в условиях, близких к термодинамическому
равновесию при температурах $T$.
На этом этапе эволюции звезды в ее недрах
будут сталкиваться уже накопившиеся стабильные ядра из главной последовательности.
Хотя средняя относительная энергия столкновения определяется температурой
$T$, всегда будет некоторое количество праматеринских ядер $(A,Z)$, энергия
столкновения которых с другими ядрами превышает $\Delta$ и открывает
канал СБР  $(A,Z)\stackrel{(\beta^-)}{\longrightarrow}(A,Z+1)$. Принимая максвелловское
распределение скоростей, для полной скорости процесса СБР ядер $(A,Z)$  c
появлением ядер $(A,Z+1)$ получим:
\begin{multline}\label{vel}
P^{(СБР)}_\beta((A,Z)\rightarrow  (A,Z+1)) = \\= n(A,Z)\sum_{(A',Z')}
{n(A',Z')\over 1+\delta_{A\,A'}\delta_{Z\,Z'}} \langle \sigma_\beta^{(col)} V_i\rangle =\\
=n(A,Z)\sum_{(A',Z')}
{n(A',Z')\over 1+\delta_{A\,A'}\delta_{Z\,Z'}}\left({8\over \pi\mu_{AA'} T^3}\right)^{1/2}\times\\
\times
\sum_{\beta_f}\int\limits_{\Delta+\Delta_f+1}^\infty\sigma_\beta^{(col)}(\beta_f)
\exp{(-\varepsilon_i/T)}\varepsilon_i\,d\varepsilon_i.
\end{multline}
Здесь $V_i$ -- начальная скорость относительного движения сталкивющихся частиц,
$n(A,Z)$ - плотность ядер $(A,Z)$,
$\mu_{AA'}=m A A'/(A+A')$. $T$ задано в энергетических единицах.
Полное сечение
процесса СБР $\sigma_\beta^{(col)}(\beta_f)$
определено формулой (\ref{sech}), если рассматриваются кулоновские столкновения
праматеринских ядер с другими ядрами среды, или формулой (\ref{ndse}), если рассматриваются
столкновения с нейтронами.

После появления в среде ядер $(A,Z+1)$
следующий этап будет связан уже с их естественным
$\bm$ - распадом, в результате которого образуются обойденные изотопы
$(A,Z+2)$.
Скорость такого распада определяется выражением:
\begin{multline}\label{vel_}
W_{\bm}((A,Z+1)\rightarrow (A,Z+2))=\\=(2\pi^3)^{-1}\sum_{\beta_f'}
\left|M^{(\beta)}(\beta_f\rightarrow\beta_f')\right|^2f_0(Z+2,E_0(\beta_f')),
\end{multline}
где  $M^{(\beta)}(\beta_f\rightarrow\beta_f')$ - ядерные матричные элементы
$\bm$ - перехода из состояния $\ket{\beta_f}$ материнского
ядра $(A,Z+1)$ в состояние $\ket{\beta_f'}$ дочернего ядра $(A,Z+2)$ (предполагается $\beta$-переход разрешенного типа),
$f_0(Z,E)$ - интегральная функция Ферми \cite{dzhel}:
\begin{eqnarray}\label{f0}
f_0(Z,E)=\int\limits_1^E F_0(Z,E_e) E_e p_e (E-E_e)^2 dE_e,
\end{eqnarray}
$E_0(\beta_f')$ - полная энергия $\beta$ - перехода
$|\beta_f\rangle\rightarrow |\beta_f'\rangle$.

Таким образом, итоговая скорость образования обойденных изотопов в звездной
среде (в единице объема в единицу времени) будет иметь вид:
\begin{multline}\label{vel_s}
P((A,Z){\to}(A,Z+2))=P_\beta^{(СБР)}((A,Z){\to}(A,Z+1))\times\\
\times W_{\beta^-}((A,Z+1){\to}(A,Z+2))
\end{multline}
Формула (\ref{vel_s}) позволяет рассчитывать
распространенности обойденных ядер
по известным плотностям нуклидов главной последовательности и характеристикам
соответствующих $\beta$-переходов.

Можно избежать расчета величины $W_{\beta^-}((A,Z+1){\to}(A,Z+2))$, если учесть, что возраст элементов $s$-процесса $\sim
10\cdot 10^9$ лет \cite{chechev}, а нижний предел среднего возраста элементов $r$-процесса $8,7\cdot 10^9$ лет
\cite{schramm73}. Тогда естественный $\beta$-распад можно считать достоверным событием и учитывать только коэффициент
ветвления (вероятность того, что произойдет именно $\beta^-$-распад, а не электронный захват или $\beta^+$-распад).

         Для всех типов $\beta$-распада предполагается
         ограничиться  только $\beta$-переходами разрешенного
         типа как наиболее интенсивными. Может возникнуть
         сомнение, возможно ли это практически для всей цепочки
         $\beta$-переходов $$(A,Z)\stackrel{\beta^-,col}
         {\longrightarrow}(A,Z+1)\stackrel{\beta^-}
         {\longrightarrow}(A,Z+2)\;?$$ Действительно, в ряде случаев
         это невозможно, если рассматривать $\beta$-переходы
         только между основными состояниями ядер, вовлеченных в
         процесс. Однако, в среде с температурами порядка ядерных
         температур (0,1 Мэв $\lesssim T \lesssim $ 1 Мэв)
         заселенными будут и возбужденные состояния ядер, и можно
         рассматривать, скажем, столкновительный $\beta^-$-распад
         не из основного состояния ядра $(A,Z)$, а из возбужденного,
         вводя поправку
         при этом на его заселенность (конечное состояние в
         процессе СБР может быть любым и по ним проводится
         суммирование сечений). Точно также может быть рассмотрен
         и естественный $\beta^-$-распад ядра $(A,Z+1)$ не из
         основного, а из
         возбужденного состояния с учетом его заселенности.
         Анализ всех $\beta$-распадных цепочек, ведущих к
         обойденным ядрам $(A,Z+2)$, показывает, что практически
         всегда могут быть найдены разрешенные $\beta$-переходы
         на всех этапах. Это  обстоятельство существенно снижает
         структурные флюктуации величин $P_\beta^{(СБР)}$ и $W_{\beta^-}$
         из-за
         степени запрещенности $\beta$-переходов, оставляя в
         качестве главного параметра величину пороговой энергии
         $\Delta$ для конкретных столкновительных $\beta$-
         распадов.

         Определенную проблему представляет расчет
         ядерных матричных элементов $\beta$-переходов. Он неизбежно
         связан с определенными модельными предположениями о
         структуре ядерных состояний, между которыми
         осуществляется переход. В настоящее время
         надежной рассчетной схемы или универсальной модели ядра,
         позволяющей получить реалистичные волновые функции
         ядерных состояний, не существует, исключая простейшие
         случаи, например, зеркальных $\beta$-переходов.
         Ситуация станет еще более сложной, если
         рассматривать $\beta$-переходы из возбужденных
         состояний (или в возбужденные). Эти ядерные состояния
         могут быть самой разной природы, и здесь при расчете
         интегралов перекрытия ядерных волновых функций на
         $\beta$-распадном операторе ошибка может быть особенно
         велика. Поэтому мы использовали два пути
         преодоления этих трудностей. Один из них предполагает там,
         где это возможно,
         извлечение экспериментальных
         матричных элементов из известных значений $\lg f_0 t$.
          Это относится не только к
         естественному $\beta^-$-распаду $(A,Z+1)\stackrel
         {\beta^-}{\longrightarrow}(A,Z+2)$, но и к
         столкновительному $\beta$-переходу $(A,Z)\stackrel
         {\beta^-,col}{\longrightarrow}(A,Z+1)$. В последнем
         случае нередко ядро $(A,Z+1)$, наряду с $\beta^-$-
         распадом, испытывает и естественный $\beta^+$-распад
         $(A,Z+1)\stackrel {\beta^+}{\longrightarrow}(A,Z)$ с
         известными значениями $\lg f_0 t$, что также позволяет
         получить экпериментальные матричные элементы. Отметим,
         что практически случаев, когда экспериментальные
         матричные элементы могли быть извлечены, не так уж мало (см. таб. 1).
         В случаях, когда из-за отсутствия экспериментальных
         данных вышеописанной процедурой воспользоваться не
         удается, применялся другой способ: величины матричных элементов
         извлекались, ориентируясь на типичные значения $\lg f_0 t$
         необлегченных $\beta$-переходов разрешенного типа.
         В подавляющем большинстве случаев они заключены в диапазоне
         $4,5 \div 5,0$. Это, конечно, приводит к некоторому
         разбросу конечных результатов, однако, сводит к минимуму
         ошибки от незнания реальной структуры ядерных состояний.
Величина $f_0 t$  фактически является универсальной константой для разрешенного
$\beta$-распада и позволяет определить из экспериментальных данных величину
ядерного матричного элемента:
\begin{eqnarray}\label{mat_sec}
\left\{\mod{M_v}^2+(g_a/g_v)^2\mod{M_a}^2\right\}={{2\pi^3 \ln 2}\over{g_v^2 f_0 t}}
\approx {{6250 \; секунд}\over{f_0 t}},
\end{eqnarray}
и таким образом получить сведения об изменении состояния ядра при $\beta$-переходе.

В таблице 1 собраны необходимые для расчетов распространенностей обойденных
изотопов значения пороговой энергии $\Delta$ и приведенного
времени жизни $f_0 t$ для столкновительного $\beta$-распада праматеринских
(по отношению к обойденным) ядер. При составлении таблицы использовались
данные из \cite{isotopes}.

%ТАБЛИЦА 1

\noindent
\begin{table}
\caption{Характеристики праматеринских ядер.}
\tabcolsep=5pt
\begin{tabular}{|c|c|c|c|c|c|}
\hline
 Праматеринское& Материнское & $\; J_i\; $&$J_f$&$\Delta + \Delta_f,$&${\rm lg} f_0t$\\
  ядро  &   ядро &&&$МэВ$&\\
 \hline
  $^{74}Ge$ & $^{74}As$ & $0^+$  &  $(1^+)$ & 2,766 &      \\
  $^{74}Ge$ & $^{74}As$ & $0^+$  &  $(1^+)$ & 2,982 &      \\
  $^{78}Se$ & $^{78}Br$ & $0^+$  &  $1^+$  & 3,5737 &   4,8   \\
  $^{80}Se$ & $^{80}Br$ & $0^+$  &  $1^+$  & 1,8703 &   4,6  \\
  $^{84}Kr$ & $^{84}Rb$ & $0^+$  &  $2^-$  & 2,68   &   9,6  \\
  $^{106}Pd$ & $^{106}Ag$ & $0^+$  &  $1^+$  & 2,983 &   4,9   \\
  $^{106}Pd$ & $^{106}Ag$ & $2^+$  &  $1^+$  & 2,471 &   5,3   \\
  $^{108}Pd$ & $^{108}Ag$ & $0^+$  &  $1^+$  & 1,921 &   4,8   \\
  $^{108}Pd$ & $^{108}Ag$ & $2^+$  &  $1^+$  & 1,487 &   5,5   \\
  $^{112}Cd$ & $^{112}In$ & $0^+$  &  $1^+$  & 2,578 &   4,7   \\
  $^{112}Cd$ & $^{112}In$ & $2^+$  &  $1^+$  & 1,961 &   5,3   \\
  $^{114}Cd$ & $^{114}In$ & $0^+$  &  $1^+$  & 1,9846 &   4,8   \\
  $^{114}Cd$ & $^{114}In$ & $2^+$  &  $1^+$  & 1,4266 &   5,3   \\
  $^{120}Sn$ & $^{120}Sb$ & $0^+$  &  $1^+$  & 2,681 &   4,5   \\
  $^{124}Te$ & $^{124}I$  & $0^+$  &  $2^-$  & 3,157 &   9,3   \\
  $^{124}Te$ & $^{124}I$  & $2^+$  &  $2^-$  & 2,555 &   7,5   \\
  $^{126}Te$ & $^{126}I$  & $0^+$  &  $2^-$  & 2,156 &   9,2   \\
  $^{126}Te$ & $^{126}I$  & $2^+$  &  $2^-$  & 1,49 &   7,4   \\
  $^{130}Xe$ & $^{130}Cs$ & $0^+$  &  $1^+$  & 3,019 &   5,1   \\
  $^{130}Xe$ & $^{130}Cs$ & $2^+$  &  $1^+$  & 2,483 &   6,4   \\
  $^{132}Xe$ & $^{132}Cs$ & $2^+$  &  $2^{(-)}$  & 1,443 &   6,0   \\
  $^{136}Ba$ & $^{136}La$ & $0^+$  &  $1^+$  & 2,87 &   4,6   \\
  $^{164}Dy$ & $^{164}Ho$ & $0^+$  &  $1^+$  & 1,0292 &   4,6   \\
  $^{164}Dy$ & $^{164}Ho$ & $2^+$  &  $1^+$  & 0,9558 &   4,9   \\
  \hline
\end{tabular}
\label{Tels}
\end{table}

Как известно, $\beta$-распад может быть запрещен или заторможен вследствие
заселенности в электронном фазовом пространстве состояний,
в которые обычно попадает конечный
электрон в результате распада. Так как в звездах большие плотности вещества,
а соответственно
и большие плотности свободных электронов, то возникает вопрос о необходимости
точного учета принципа Паули. Для ответа на этот вопрос необходимо сравнить
химический потенциал (энергию Ферми) при нулевой температуре  с
величиной $kT$ при соответствующих плотностях.

Химический потенциал $\mu_0$ в релятивистском случае определяется формулой:
\begin{eqnarray}\label{him}
\mu_0=\sqrt{p_0^2 c^2+m^2 c^4}-mc^2,
\end{eqnarray}
где $p_0$ -- импульс Ферми, для которого согласно \cite{LL_stat} имеем:
\begin{eqnarray}\label{p_0}
p_0=({3\pi^2 \hbar^3 n})^{1/3},
\end{eqnarray}
($n$ -- электронная плотность).
Из (\ref{him}) с учетом (\ref{p_0}) получим
\begin{eqnarray}\label{him2}
\mu_0=m c^2\left \{\sqrt{1+ {{({3\pi^2 \hbar^3 n})^{2/3}}\over{m^2 c^2}} }-1\right \},
\end{eqnarray}
или с учетом величин констант:
\begin{eqnarray}\label{him3}
\mu_0=m c^2\left \{\sqrt{1+ 1.55\cdot 10^{-20} n_0^{2/3}}-1\right \},
\end{eqnarray}
где $n_0$ -- электронная плотность в единицах $см^{-3}$.  Отсюда  видно, что при рассматриваемых нами температурах $\mu_0
< kT$, если $n_0\sim 10^{29} см^{-3}$. При $\mu_0 < kT$, величина $\mu (T)$ будет уже отрицательной \cite{kubo}, и
пренебрежение принципом Паули приведет к завышению вероятности \be-распада не более, чем  в два раза. Имея в виду
неопределенность в величинах многих других расчетных параметров нет необходимости отягощать расчеты точным учетом
принципа Паули, по крайней мере, на данном этапе исследования рассматриваемых моделей.


При определении коэффициентов ветвления для естественного \be-рас\-па\-да
материнских ядер возникает вопрос учета их возможной
ионизации, что может существенно увеличить выход обойденных
изотопов.
Рассмотрим отношение числа ионов с пустой $K$-оболочкой, двумя
свободными электронами и $(Z-2)$ электронами, которые могут быть как свободными,
так и связанными,
к числу ионов с заполненной $K$-оболочкой (остальные $(Z-2)$ электрона также
могут быть как свободными, так и связанными).
Это есть отношение сумм по всем возможным состояниям первой и второй систем,
умноженное на больцмановский фактор, т.е.
\begin{eqnarray}\label{i1}
{{(2I+1)^2 \sum_m \int \int {{d\vec p_1 d\vec p_2 d\vec r_1 d\vec r_2}\over{(2\pi \hbar)^6}} e^{-{\varepsilon_m \over kT}
-{p_1^2 \over 2m_ekT}-{p_2^2 \over 2m_ekT}} e^{-{\chi_{1m} + \chi_{2m} \over kT}} }\over{\sum_{m'} e^{-{\varepsilon_{m'}\over kT}}}},
\end{eqnarray}
где $\varepsilon_{m}$ и $\varepsilon_{m'}$ -- энергия состояний соответственно
иона с заполненной $K$-оболочкой и иона с пустой $K$-оболочкой,  $\chi$ -- энергия связи
$K$-электрона, $I=1/2$ -- спин электрона,
$\vec r$ и  $\vec p$ -- координата и импульс электрона.

Так как набор состояний у первой и второй систем фактически один и тот же
(перебираем состояния $(Z-2)$ электронов), то
с достаточной для оценки точностью
можно считать
\begin{eqnarray}\label{i01}
\sum_m e^{-{\varepsilon_m \over kT} e^{-{\chi_{1m} + \chi_{2m} \over kT}} }
\approx e^{-{(\chi_1 + \chi_2) \over kT}} } \sum_{m'} e^{-{\varepsilon_{m'} \over kT}.
\end{eqnarray}
В результате получается выражение, аналогичное формуле Со\-ха-Боль\-ц\-ма\-на (\cite{aller}, \cite{menzel}):
\begin{eqnarray}\label{i2}
{n_I n_e^2 \over n_A}={{4(2\pi m_e kT )^3}\over{(2\pi \hbar)^6}} e^{-{\chi_1 + \chi_2 \over kT}},
\end{eqnarray}
где $n_I$, $n_e$ и $n_A$ --- плотности ионизованных атомов, электронов и нейтральных атомов
соответственно. Положив $n_e= n_I$, получаем:
\begin{eqnarray}\label{i3}
n_I= n_A^{1/3} {{ m_e kT }\over{2\pi \hbar^2}} e^{-{\chi_1 + \chi_2 \over 3kT}},
\end{eqnarray}
или
\begin{eqnarray}\label{i4}
n_I= 1,8\cdot 10^{10} T n_A^{1/3}  e^{-{1680(\chi_1 + \chi_2) \over T}},
\end{eqnarray}
если энергия выражена в электрон-вольтах,  температура в кельвинах,
$n_I$, $n_A$ -- в $см^{-3}$.
%Так как энергия ионизации $k$-электронов будет значительно меньше энергии связи
%$k$-электронов в гелееподобном атоме с зарядом $Z$, а последняя при $Z<80$
%$E_{св} \sim 200 \; кэВ$, то экспоненциальным множителем в оценочных расчетах
%можно пренебречь.


К вышеприведенному выводу формулы для оценки $n_I/n_A$ следует сделать следующее замечание.
При расчете статистической суммы при интегрировании в фазовом пространстве по
импульсам электронов ввиду высокой температуры существенный вклад будут давать
состояния с энергией до $m_ec^2$. Казалось бы обязательно надо пользоваться релятивистской
формулой, т.е.  интеграл по импульсам брать в виде:
\begin{eqnarray}\label{i5}
{\cal I}=\int e^{-{{ \varepsilon -m_ec^2}\over kT}} {{d\vec p}\over{(2\pi \hbar)^3}},
\end{eqnarray}
где $\varepsilon=\sqrt{\vec p^2 c^2 + m_e^2c^4}$.
Этот интеграл берется элементарно и дает:
\begin{eqnarray}\label{i6}
{\cal I}={4\pi \over {(2\pi \hbar)^3} } (m_ec)^3 {kT\over {m_ec^2}}\left [K_0 \left ({m_ec^2\over kT}\right )+
2 {kT\over {m_ec^2}} K_1 \left ({m_ec^2\over kT}\right ) \right ] e^{{m_ec^2\over kT}},
\end{eqnarray}
где $K_0(x)$ и $K_1(x)$ -- функции Бесселя мнимого аргумента (или модифицированные
функции Бесселя)  порядка 0 и 1 соответственно.

Рассмотрим предельные случаи.

При ${m_ec^2\over kT} \gg 1$ $K_0(x)$ и $K_1(x)$ имеют асимптотику \cite{gr_r}:
\begin{eqnarray}\label{i7}
K_0 \left ({m_ec^2\over kT}\right )\approx K_1\left ({m_ec^2\over kT}\right) \approx \sqrt{{\pi\over 2}\cdot {{kT\over {m_ec^2}}}} e^{-{m_ec^2\over kT}},
\end{eqnarray}
и для ${\cal I}$ имеем в точности нерелятивистский результат:
\begin{equation}\label{i8}
{\cal I}={1\over \hbar^3}\left({m_ekT\over 2\pi}\right)^{3/2}
\end{equation}

В другом пределе ${m_ec^2\over kT} \ll 1$:
\begin{equation}\label{i9}
K_0\left ({m_ec^2\over kT}\right ) \approx \ln{2kT \over m_ec^2},
\qquad K_1\left ({m_ec^2\over kT}\right )\approx {kT \over m_ec^2},
\end{equation}
и для оценки интеграла получаем:
\begin{equation}\label{i10}
{\cal I}=8 \pi\left({kT\over 2\pi\hbar c}\right)^3.
\end{equation}

Как следует из формул (\ref{i8}) и (\ref{i10})  в предельных случаях ${\cal I}$  имеет неодинаковые значения,
но оказывается, что в области $kT=m_ec^2$ они отличаются
менее, чем в два раза. Однако, поскольку наши расчеты
претендуют на оценку  только
порядка величин физических характеристик,  можно пользоваться нерелятивистской
формулой (\ref{i1}), не только при
$kT\ll m_ec^2$, но и при менее жестком условии
$kT\lesssim m_ec^2$, что имеет место в нашем случае.

Как показали оценки, учет возможной ионизации материнских ядер
может увелить вероятность естественного $\beta^-$-распада (по сравнению с
$K$-захватом) в пределах $10\div 100\,\%$.

Экспериментальные исследования химического состава Солнца показали, что
он очень сходен с составом Земли и метеоритов. В то же время самые старые из
известных звезд - субкарлики, образовавшиеся на наиболее  ранних стадиях истории
Галактики, - содержат меньше тяжелых элементов, чем Солнце. Это позволяет
предположить, что значительное количество тяжелых элементов, которое
наблюдается в молодых звездах, синтезировалось вскоре после образования
Галактики в недрах массивных звезд первого поколения в процессе их эволюции,
заканчивающейся взрывами сверхновых. При этом в межзвездную среду было
выброшено значительное количество их вещества, которое в дальнейшем послужило
материалом для образования звезд второго поколения, в том числе и Солнца.
В Солнце заключена большая часть массы Солнечной системы, и его вещество,
по-видимому, не подверглось химической дифференциации. Поэтому, когда говорят
о распространенностях элементов в Солнечной системе, то в действительности имеют
в виду Солнце, принимая содержания элементов на Солнце и в первичной солнечной
туманности одинаковыми. К сожалению, спектроскопический анализ содержания
элементов в солнечной фотосфере не обладает столь большой точностью, как
химический и радиохимический анализ твердых веществ. Практически это означает,
что содержания элементов в метеоритах должны выбираться в качестве стандарта при
систематизации распространенностей большинства элементов.

В таблице \ref{Tels2} каждому нуклиду, являющемуся праматеринским ядром в нашей столкновительной модели, сопоставляется
определенный процесс его нуклеосинтеза, который вносит наибольший вклад в его космическую распространенность, а также
приведены распространенности таких ядер в первичной солнечной туманности по отношению к кремнию $(10^6)$ и аналогичные
распространенности обойденных  изотопов. При составлении таблицы использовались данные из \cite{iader}.
%ТАБЛИЦА 2
\noindent
\begin{table}
\caption{Распространенности нуклидов.}
\tabcolsep=5pt
\begin{tabular}{|c|c|c|c|c|}
\hline
 Праматерин-& Процесс & Распростра-&Обойденное&Распростра-\\
 ское ядро  &         & ненность&     ядро     &ненность\\
 \hline
  $^{74}Ge$ & $s$     & 42,8  & $^{74}Se$ & 0,58   \\
  $^{78}Se$ & $s$,$r$ & 15,8  & $^{78}Kr$ & 0,146   \\
  $^{80}Se$ & $s$,$r$ & 33,4  & $^{80}Kr$ & 0,94   \\
  $^{84}Kr$ & $s$,$r$ & 23,5  & $^{84}Sr$ & 0,128   \\
  $^{92}Zr$ & $s$,$r$ & 2,05  & $^{92}Mo$ & 0,634   \\
  $^{94}Zr$ & $s$,$r$ & 2,09  & $^{94}Mo$ & 0,361   \\
  $^{96}Mo$ & $s$     & 0,661  & $^{96}Ru$ & 0,105   \\
  $^{98}Mo$ & $s$,$r$ & 0,951  & $^{98}Ru$ & 0,0355   \\
  $^{102}Ru$ & $s$,$r$ & 0,601  & $^{102}Pd$ & 0,0125   \\
  $^{106}Pd$ & $s$,$r$ & 0,355  & $^{106}Cd$ & 0,0188   \\
  $^{108}Pd$ & $s$,$r$ & 0,347  & $^{108}Cd$ & 0,0136   \\
  $^{112}Cd$ & $s$,$r$ & 0,373  & $^{112}Sn$ & 0,0355   \\
  $^{114}Cd$ & $s$,$r$ & 0,447  & $^{114}Sn$ & 0,0244   \\
  $^{120}Sn$ & $s$,$r$ & 1,22  & $^{120}Te$ &  0,0058  \\
  $^{124}Te$ & $s$,$r$ & 0,454  & $^{124}Xe$ & 0,0074   \\
  $^{126}Te$ & $s$,$r$ & 1,22  & $^{126}Xe$ &  0,0067  \\
  $^{130}Xe$ & $s$     & 0,25  & $^{130}Ba$ &  0,00485  \\
  $^{132}Xe$ & $s$,$r$ & 1,52  & $^{132}Ba$ &  0,00466  \\
  $^{136}Ba$ & $s$     & 0,375  & $^{136}Ce$ & 0,0023   \\
  $^{138}Ba$ & $s$,$r$ & 3,44  & $^{138}Ce$ &  0,003  \\
  $^{144}Nd$ & $s$,$r$ & 0,188  & $^{144}Sm$ & 0,00742   \\
  $^{152}Sm$ & $r$     & 0,0641  & $^{152}Gd$ & 0,00084   \\
  $^{156}Gd$ & $s$,$r$ & 0,0860  & $^{156}Dy$ & 0,000193   \\
  \hline
\end{tabular}
\label{Tels2}
\end{table}


\noindent
\begin{table}
\tabcolsep=5pt
\begin{tabular}{|c|c|c|c|c|}
\hline
 Праматерин-& Процесс & Распростра-&Обойденное&Распростра-\\
 ское ядро  &         & ненность&     ядро     &ненность\\
 \hline
  $^{158}Gd$ & $s$,$r$ & 0,104  & $^{158}Dy$ &  0,000334  \\
  $^{162}Dy$ & $s$,$r$ & 0,0945  & $^{162}Er$ & 0,000313   \\
  $^{164}Dy$ & $s$,$r$ & 0,104  & $^{164}Er$ &  0,00359  \\
  $^{168}Er$ & $s$,$r$ & 0,0623  & $^{168}Yb$ & 0,00027   \\
  $^{174}Yb$ & $s$,$r$ & 0,0637  & $^{174}Hf$ & 0,00031   \\
  $^{180}Hf$ & $s$,$r$ & 0,0599  & $^{180}W$ &  0,000202  \\
  $^{184}W$ & $s$,$r$ &  0,0919 & $^{184}Os$ &  0,000124  \\
  $^{190}Os$ & $s$,$r$ & 0,182  & $^{190}Pt$ &  0,000179  \\
  $^{196}Pt$ & $s$,$r$ & 0,357  & $^{196}Hg$ &  0,00062  \\
  \hline
\end{tabular}
\end{table}


 Результаты расчетов распространенностей
 обойденных изотопов по формулам (\ref{sech}), (\ref{ndse}) и (\ref{vel})
 в сравнении с экспериментальными данными представлены на рис.\ref{RCUL},\ref{RNEI}
(рассматривались процессы столкновительного $\beta$-распада, стимулированные ядро-ядерными и нейтрон-ядерными
столкновениями соответственно). При их получении вводились следующие упрощения. Во-первых, чтобы исключить такие
параметры, как время жизни и объем звезды в квазиравновесном состоянии, рассчитывались относительные распространенности
обойденных изотопов. Во-вторых, принималось равномерное распределение элементов в веществе звезды (соответствующие
плотности брались из \cite{iader}). И, наконец, в-третьих, не проводился прямой расчет $\beta$-распадных матричных
элементов, который  неизбежно был бы связан с модельными предположениями о структуре соответствующих ядерных состояний,
что могло привести к ошибочным результатам (этот вопрос был обсужден выше), а  находились численные величины необходимых
матричных элементов по экспериментальным и типичным ($\lg f_0t=4,5\div 5,0$ для необлегченных $\beta$-переходов
разрешенного типа) значениям приведенных времен жизни (эта процедура была описана выше).

\begin{figure}
\vspace{18 true cm}
\caption{{ Относительная распространенность $N=P(A,Z\to A,Z+2)/P({}^{108}Pd\to {}^{108}Cd)$
обойденных ядер, образованных согласно столкновительной модели (рассматривались
кулоновские ядро-ядерные столкновения).  Сплошная линия -- наблюдаемые значения, штрих-пунктирная -- расчетные.}}
\label{RCUL}
\end{figure}

\begin{figure}
\vspace{18 true cm}
\caption{{ Относительная распространенность $N=P(A,Z\to A,Z+2)/P({}^{108}Pd\to {}^{108}Cd)$
обойденных ядер, образованных согласно столкновительной модели
(рассматривались нейтрон-ядерные столкновения).
Сплошная линия -- наблюдаемые значения, штрих-пунктирная -- расчетные.}}
\label{RNEI}
\end{figure}


Как видно из рис. \ref{RCUL} и \ref{RNEI}, теоретическая кривая в главном воспроизводит
экспериментальную зависимость относительной распространенности обойденных
изотопов от массового числа. В то же время в отдельных случаях выход обойденных
ядер в рамках принятой модели недостаточен.
Как правило, это имеет место в тех случаях, где пороговая энергия $\Delta$ относительно велика,
а от величины $\Delta$ сильно зависит сечение $\sigma_\beta^{(col)}(\beta_f)$.
Если принять за факт реальность механизма СБР в данном процессе
и исключить существенное изменение ядерных матричных элементов
при введении структурных поправок, то можно сделать вывод, что данная
стадия эволюции звезды для процесса синтеза таких  изотопов малосущественна.
Не исключено, что на заключительном этапе термоядерной эволюции, когда
происходит дальнейший разогрев вещества и взрыв сверхновой, процесс их синтеза таких
изотопов пойдет более интенсивно, чем для изотопов с $\Delta\lesssim 2 Мэв$.

Практическая значимость рассматриваемых процессов определяется сравнительными
величинами кулоновского и ядерного сечений процесса столкновительного
$\beta$-распада, с одной стороны, и соотношением плотностей легких ядер и
нейтронов в среде, с другой. Несмотря на то, что сечение процесса СБР,
стимулированного нейтрон-ядерными столкновениями примерно, на $7 \div 12$
порядков (в зависимости от энергии столкновения) больше соответствующих
сечений $\beta$-распада, инициированного кулоновскими  ядро-ядерными столкновениями, он
на квазиравновесной стадии эволюции звездного вещества, по-видимому,
проигрывает из-за
маленькой реальной плотности нейтронов ($\sim 10^7 \; нейтрон \cdot см^{-3}$)
по сравнению с плотностью легких ядер, таких как водород и гелий. Так, если
плотность среды $\sim 10^3 \; г/см^3$ (например, в водородо-гелиевых оболочках),
то плотность протонов будет $\sim 10^{27} \; см^{-3}$.

Возникает вопрос: возможны ли столкновения тяжелых праматеринских ядер, которые
вероятнее всего находятся в ядре звезды, с легкими ядрами, основное количество
которых вытеснено в оболочку? Да, и это объясняется в ряде звездных моделей,
например, в модели Бекера и Ибена \cite{becker}. Согласно этой модели  во внешней
конвективной оболочке происходят вспышки, в промежутках между которыми она опускается
вниз (к центру звезды) и к ней подмешивается некоторое количество синтезированного (в нашем случае
в $s$-процессе) вещества. Окончательный результат такой же, как если бы
вещество непосредственно было смешано с оболочкой. Аналогичная модель используется
также в $s$-процессе, когда рассматривается синтез тяжелых ядер путем захвата
нейтронов, образованных в гелиевых оболочках, элементами железного пика,
находящимися в ядре звезды.
На рис.\ref{SVT}  представлена зависимость величины $\left\langle\sigma_\beta^{(col)} V_i\right\rangle$
от температуры.

\begin{figure}
\vspace{18 true cm}
\caption{{
Зависимость величины $\left\langle\sigma_\beta^{(col)}(\beta_f) V\right\rangle$
от температуры в случае кулоновских ядро-ядерных столкновений с водородом ядер:
(1) $^{110}Pd\,(\Delta=0,879\, МэВ)$, (2) $^{80}Se\,(\Delta=1,87\, МэВ)$, (3) $^{120}Sn\,(\Delta=2,68\, МэВ)$,
(4) $^{136}Ba\,(\Delta=2,87\, МэВ)$, (5) $^{78}Se\,(\Delta=3,5737\, МэВ)$.}}
\label{SVT}
\end{figure}

Для получения абсолютных значений распространенностей обойденных изотопов
(по отношению к распространенностям праматеринских ядер)
необходимо знать временные протяженности заключительных этапов звездной
эволюции, предшествующих гравитационному коллапсу, характерные
температуры и плотности вещества на этих стадиях. К сожалению, эти данные довольно неопределенны.
Однако, так как вероятность в единицу времени процессов
СБР невелика, у звезды на квазиравновесной стадии, скорее всего будет
недостаточно времени для их реализации, поэтому возникает вопрос исследования
альтернативных механизмов образования обойденных изотопов.






\section{Нуклеосинтез обойденных ядер в $\beta$-процессе, индуцированном
электро- \\ магнитным излучением.}

В связи с проблемой происхождения обойденных изотопов исследуем
альтернативный механизм, известный ранее, но пока не рассматривавшийся в приложении к ней.
Это $\beta$-распад стабильных ядер, инициируемый
поглощением теплового электромагнитного излучения:
\begin{eqnarray}\label{gamma}
\gamma+(A,Z) \to (A,Z+1)+e^- + \tilde \nu_e.
\end{eqnarray}
Введем обозначение:
\begin{eqnarray}\label{D_gamma}
\Delta_\gamma= M(A,Z+1)c^2 + m_e c^2 - M(A,Z) c^2,
\end{eqnarray}
где $M(A,Z)$ -- масса ядра (A,Z).
В интересующем нас эндотермическом случае, как показал анализ \cite{shaw},
основной вклад в реакцию вносит процесс,
когда фотон $\gamma$ рождает в кулоновском поле ядра виртуальную
электрон-позитронную пару и позитрон поглощается ядром с испусканием
антинейтрино. Диаграмма этого процесса представлена на рис.~\ref{DFOT}.
\begin{figure}
\vspace{5 true cm}
\caption{Основная диаграмма эндотермического процесса  \be-распада, индуцированного
электромагнитным излучением.}
\label{DFOT}
\end{figure}

Скорость $R^{(\gamma)}
(T,\Delta_\gamma)$ такого индуцированного электромагнитным излучением
$\beta$-распада можно вычислить по формуле \cite{shaw} (используется
система единиц  $m_e = \hbar = c = 1$):
\begin{eqnarray}\label{R_gamma}
R^{(\gamma)}(T,\Delta_\gamma)={{\ln 2}\over{\pi}}{{\alpha_e}\over{ft}}
\int\limits_{\Delta_\gamma}^\infty {{d\omega}\over{\omega}}{{G(\omega, \Delta_\gamma)}
\over{\exp({\omega/(kT)}-1)}},
\end{eqnarray}
где
\begin{eqnarray}\label{G}
G(\omega, \Delta_\gamma)&=&\int\limits_{1}^{\omega - \Delta_\gamma + 1}
(\omega-E- \Delta_\gamma +1)^2[2(\omega-E)(E^2-1)^{1/2} +\nonumber\\
&&+ (\omega^2-
2\omega E+2E^2)\ln(E+(E^2-1)^{1/2})]\,dE,
\end{eqnarray}
а величина $ft$ определяет ядерный матричный элемент перехода из состояния
$(A,Z)$ в состояние $(A,Z+1)$.

Результаты расчетов скорости $R^{(\gamma)}$  индуцированного
электромагнитным излучением $\beta$-распада
как функции температуры $T$ и
пороговой энергии $\Delta_\gamma$
представлены на рис. \ref{SCOR}.

\begin{figure}
\vspace{18 true cm}
\caption{{Зависимость скорости $R^{(\gamma)}$  индуцированного
электромагнитным излучением $\beta$-распада
от температуры $T$ и
пороговой энергии $\Delta_\gamma$:
$\Delta_\gamma=1\, МэВ$ (1), $\Delta_\gamma=2\, МэВ$ (2),
$\Delta_\gamma=2,5\, МэВ$ (3), $\Delta_\gamma=3,5\, МэВ$ (4).
$f_0t=4,5$.
}}
\label{SCOR}
\end{figure}
По-видимому,  этот процесс может играть заметную роль на квазиравновесной
стадии звездной эволюции и протекает скорее всего в красных гигантах
и сверхгигантах после образования там ядер $s$-процесса при $T>10^9 K$.
Как известно, красные гиганты и сверхгиганты могут иметь ядра из относительно тяжелых
элементов в районе железного максимума,
но время жизни таких звезд крайне мало -- всего $\sim 10^3$ лет.
Тем не менее, этого, по-видимому, достаточно для реализации рассматриваемого
процесса. Так, если в звезде при плотности вещества не больше, чем $10^5\; г/см^3$
в течение, по крайней мере, $100$ лет будет
температура $T\sim 2\cdot 10^9\; K$ и(или) в течение хотя бы $10$ лет --- температура
$T\sim 3\cdot 10^9\; K$,
то получается хорошее согласие экспериментальных и рассчитанных
величин не только относительных, но уже и абсолютных распространенностей
обойденных изотопов (см. рис. \ref{abs} ). При этом следует отметить,
что приведенные выше условия
вполне реальны, если рассматривать нуклеосинтез, например, в горящих
кислородных оболочках
(горение кислорода происходит при температуре
$T>10^9 \; K$, а горение кремния при $T\sim 3\cdot 10^9 \div 4\cdot 10^9\; K $).
Все это дает основание рассматривать процесс
$\beta$-распада, инициируемый
поглощением теплового электромагнитного излучения,
как
лидирующий среди возможных механизмов образования обойденных ядер.


\begin{figure}
\vspace{18 true cm}
\caption{{ Абсолютная распространенность обойденных ядер $P(A,Z+2)$
(по отношению к праматеринским ядрам $P(A,Z)$), образованных
на основе процесса
$\beta$-распада, индуцированного электромагнитным излучением.}}
\label{abs}
\end{figure}





\section{Образование обойденных ядер на \\ основе $\beta$-распада из
возбужден- \\ ного состояния ядра.}

Рассмотрим еще один возможный альтернативный механизм процесса синтеза обойденных ядер, в котором $\beta$-распад
праматеринского ядра $(A,Z)$ происходит из возбужденного состояния.
Хотя характерные скорости $\beta$-распадов значительно уменьшаются при \be-переходах
с возбужденных уровней энергии ядер, в случае, когда обычный $\beta$-распад
запрещен законом сохранения энергии, такой процесс может быть существенным.
Физический механизм $\beta$-распада из термически возбужденного состояния
подразумевает двухступенчатость процесса (см. рис. \ref{VOZ}).
В нем на первом этапе возбужденное
состояние материнского ядра вызывается обменом энергией между веществом и излучением в условиях
статистического равновесия, а на втором -- происходит $\beta$-переход с возбужденного состояния.

\begin{figure}
\vspace{18 true cm}
\caption{Схема \be-распада из возбужденного состояния ядра.}
\label{VOZ}
\end{figure}


Энергетическим критерием для осуществления такого $\beta$-распада является
условие:
\begin{eqnarray}\label{y_v}
M(A,Z) c^2 > M(A,Z+1) c^2 + m_e c^2 - E^*,
\end{eqnarray}
где  $M(A,Z)$ и $M(A,Z+1)$ -- массы материнского  и дочернего ядер в основном
состоянии соответственно, $m_e$ -- масса электрона,
$E^*$ -- энергия возбужденного состояния материнского ядра, отсчитанная от основного.

Возбужденные уровни могут заполняться в условиях термодинамического равновесия при высоких температурах в недрах звехды.
Скорость распада с указанных возбужденных уровней $R_e$ может быть найдена по формуле \cite{l86}:
\begin{eqnarray}\label{R_e}
R_e={{1}\over{P(T)}}\sum_i \sum_j \lambda_{ij}(2J_i+1) \exp{\left(-{{E_i}\over{kT}}\right)},
\end{eqnarray}
где $\lambda_{ij}$ -- скорость $\beta$-распада, при котором имеет место переход из
родительского состояния $(i)$ в дочернее состояние $(j)$, $E_i$ и $J_i$ --
энергия возбуждения и спин родительского состояния $(i)$. Предполагается, что
состояния $(i)$ заполнены в соответствии с распределением Больцмана, а
ядерная статистическая сумма  $P(T)$ определяется соотношением:
\begin{eqnarray}\label{P(T)}
P(T)=\sum_i (2J_i+1) \exp{\left(-{{E_i}\over{kT}}\right)}.
\end{eqnarray}
Статистическую сумму часто принимают равной статистическому весу основного
состояния $2J_0+1$, где $J_0$ -- спин основного состояния с энергией $E_0=0$.

Для расчета скоростей $\beta$-распада из возбужденных состояний должны быть известны
распределения по уровням и параметры уровней рассматриваемых ядер. Такая
информацию  может быть получена из \cite{isotopes}.

Процесс \be-распада из возбужденного состояния может быть достаточно эффективен
при пороговых энергиях $\Delta=1\div 3 Мэв$ только при температуре
$T\sim 5\cdot 10^9 K$. Как показал анализ, достаточно высокие температуры звездной среды (в том числе
и $T\sim 5\cdot 10^9 K$), достигаемые в процессе звездной эволюции,
 наличие у соответствующих материнских ядер возбужденных состояний с подходящими квантовыми характеристиками
обеспечивают  возможность обсуждаемого  механизма. Расчеты  относительных распространенностей оказываются в  хорошем
согласии с экспериментальными данными (см. рис. \ref{RVOZ}). Однако, эффективность процесса сильно зависит от временной
протяженности статистически равновесного этапа с $T\sim 5\cdot 10^9 K$, а она вряд ли велика. При охлаждении вещества в
сравнении с \be-распадом намного более вероятен оказывается $\gamma$-переход из возбужденного состояния в нижележащие и
основное \cite{solov}, что существенно уменьшит выход обойденных изотопов. Тем не менее возможный вклад процесса
\be-распада из возбужденных состояний праматеринских ядер не исключается.

\begin{figure}
\vspace{18 true cm}
\caption{{ Относительная распространенность $N=P(A,Z\to A,Z+2)/P({}^{108}Pd\to {}^{108}Cd)$
обойденных ядер, образованных на основе \be-распада из возбужденного состояния ядра.}}
\label{RVOZ}
\end{figure}








\section{Роль катастрофической стадии эволюции звезд в образовании
обойденных изотопов.}

Рассмотрим роль конечных стадий звездной эволюции в образовании
химических элементов. У звезд с  ${\cal M} >8 {\cal M}_{\bigodot}$
могут, в принципе, в центральной области последовательно
выгореть кислород, неон, магний, сера, кремний, после чего образуется
ядро, состоящее из элементов группы железа -- от $Sc$ до $Ni$.
Звезда приобретает структуру, подобную ``луковице'': ``железное'' ядро
окружено многочисленными слоями из продуктов ядерного горения
на предыдущих стадиях. После образования ``железного'' ядра, а в некоторых
случаях и раньше, звезда теряет гидродинамическую устойчивость --
происходит гравитационный коллапс.

Гравитационный коллапс звезды --- это катастрофически быстрое ее сжатие
под действием собственных сил тяготения --- может произойти
после прекращения в центральной области звезды термоядерных реакций.
Термоядерные реакции служат источником энергии звезды и обеспечивают
в ней гидростатическое и тепловое равновесие вплоть до образования
в ее центральной области атомных ядер группы железа.
Эти ядра имеют наибольшую энергию связи на нуклон, так что синтез
ядер более тяжелых, чем ядра железа, уже не сопровождается
выделением энергии, а наоборот требует затрат энергии.
Лишенная с этого момента термоядерных источников
энергии звезда уже не может скомпенсировать потери энергии
во внешнее пространство, чрезвычайно возрастающие к концу
``термоядерного'' этапа эволюции.
К обычным потерям энергии с поверхности звезды (испусканию фотонов
фотосферой звезды) прибавляются объемные потери энергии,
обусловленные интенсивным излучением нейтрино и антинейтрино
центральной областью звезды.

Нескомпенсированные потери энергии нарушают равновесие звезды.
Создаются условия для сжатия ее в центральной области
под действием собственных сил тяготения.
Звезда расходует теперь гравитационную энергию, выделяющуюся
при сжатии.
Температура в сжимающейся звезде возрастает до $(5-10)\cdot 10^9\; K$.
В результате коллапса достигаются плотности $\rho\sim 10^{12} г/см^3$,
при которых энергетически выгодна нейтронизация вещества. У звезд с
${\cal M} < 2 {\cal M}_{\bigodot}$ давление вырожденного газа нейтронов может
противостоять тяготению. В этом случае образуется нейтронная звезда.
При ${\cal M} >2 {\cal M}_{\bigodot}$ коллапс неограничен и звезда превращается
в черную дыру. При остановке коллапса у границы нейтронной звезды возникает
ударная волна, которая, распространяясь наружу, вызывает сброс оболочки.

При взрывах сверхновых происходит синтез тяжелых элементов, которые затем
выбрасываются в межзвездное пространство вместе с элементами, синтезированными
в ходе предшествующей эволюции. Это определяет важнейшее космогоническое
значение сверхновых звезд.

Вспышки сверхновых -- процессы динамические, и расчеты распространенностей
обойденных изотопов и, вообще, химических элементов нужно проводить
с зависящими от времени температурой
и концентрацией нейтронов. Однако целый комплекс процессов, сопровождающих
термоядерные взрывы в ядрах и гравитационный коллапс, еще не до конца ясен и
требует дальнейшего изучения. Это -- кинетика ядерных реакций и догорание остатков
ядерного топлива, которое, в принципе, может остановить коллапс, перенос энергии,
нейтринные процессы, роль магнитных процессов и вращения, механизмы передачи энергии
от ядра к оболочке. До сих пор не решен вопрос, при каком именно астрофизическом
явлении происходит образование ядер $r$-процесса.
Поэтому провести строгий динамический расчет процесса столкновительного
$\beta$-распада,
стимулированного нейтрон-ядерными столкновениями, пока не представляется
возможным, хотя он был бы очень интересен, так как грубые оценочные расчеты
указывают на возможную эффективность такого физического явления на конечной стадии
эволюции. В частности, при высоких плотностях потока нейтронов $I\sim 10^{40}
нейтрон\cdot см^{-2}с^{-1}$ и времени облучения $\sim 10\div 100 \; секунд$
процесс столкновительного $\beta$-распада способен дать выход обойденных
изотопов, близкий к экспериментальному.


Сформулируем основные результаты, полученные в данной главе.


Разработана модель синтеза обойденных ядер в звездном веществе на
квазиравновесной стадии эволюции, основанная на явлении столкновительного
$\beta$-распада $\beta$-стабильных ядер.
Показано, что она качественно, а в ряде случаев и количественно, способна
воспроизвести сильно нерегулярный ход кривой относительной распространенности
обойденных ядер, что можно расценивать как косвенное свидетельство в пользу
реальности явления столкновительного $\beta$-распада стабильных ядер.


Получено хорошее согласие абсолютных распространенностей обойденных ядер,
образованных в результате  $\beta$-распада праматеринского  ядра,
индуцированного электромагнитным излучением, что дает основание предполагать,
что такой процесс может быть определяющим при синтезе обойденных ядер в
звездном веществе.


Показано, что при определенных условиях  заметный вклад в синтез
обойденных изотопов может также вносить $\beta$-распад из возбужденных
состояний праматеринских ядер.


Приведены аргументы в пользу предположения, что на катастрофической стадии эволюции звезд
эффективным
в образовании обойденных нуклидов может оказаться
$\beta$-распад, стимулированный нейтрон-ядерными столкновениями.

\enlargethispage{\baselineskip}

Основные результаты, полученные в данной главе, опубликованы в работах
\cite{K1, K2, K3, K4}, \cite{K5, K6, K7} и \cite{K9}.



\chapter*{Заключение}
\addcontentsline{toc}{chapter}{Заключение}



Сформулируем основные результаты, полученные в настоящей работе.

\begin{itemize}

\item[1.]

Построена теория
процесса столкновительного $\beta$-распада стабильного ядра, инициированного
ядро-ядерными кулоновскими столкновениями.

Расчеты полных сечений этого процесса, выполненные с точными кулоновскими
функциями, показали, что
в области столкновительных энергий, сравнимых по величине с пороговой
энергией $\Delta$,
приближение плоских волн дает сильно завышенные результаты. Эти расхождения
тем заметнее, чем больше зарядовое число столкновительного партнера.

В случае, когда зарядовое число столкновительного партнера $Z'\gg 1$, выход
продуктов реакции столкновительного $\beta$-распада, инициированного
ядро-ядерными столкновениями, можно считать пренебрежимо малым.


\item[2.]

Разработана модель синтеза обойденных ядер в звездном веществе на
квазиравновесной стадии эволюции, основанная на явлении столкновительного
$\beta$-распада $\beta$-стабильных ядер.

Показано, что она качественно, а в ряде случаев и количественно, способна
воспроизвести сильно нерегулярный ход кривой относительной распространенности
обойденных ядер, что можно расценивать как косвенное свидетельство в пользу
реальности явления столкновительного $\beta$-распада стабильных ядер.

\item[3.]

С использованием мультипольных разложений по относительной координате оператора
перехода и волновых функций непрерывного спектра,
получено замкнутое выражение для полного сечения процесса столкновительного
$\beta$-распада, стимулированного нейтрон-ядерными столкновениями.
Волновая функция относительного движения в нуклон-ядерной системе находится из
уравнения Шредингера с оптическим потенциалом.

Показано, что при нейтрон-ядерных столкновениях величины полных сечений
существенно увеличиваются по сравнению с кулоновскими столкновениями
стабильных ядер с протонами тех же энергий (примерно на $7 \div 12$ порядков
в зависимости от энергии столкновения).

\item[4.]

Выявлена чувствительность полных сечений процесса столкновительного
$\beta$-распада, стимулированного как кулоновскими ядро-ядерными,
так и сильными нейтрон-ядерными столкновениями, к величине $\Delta$
при малых энергиях относительного движения $\varepsilon_i$.
При $\varepsilon_i \gg \Delta$, как и следовало ожидать, различие между
величинами сечений (для разных $\Delta$) становится незначительным.


\item[5.]

Получено хорошее согласие абсолютных распространенностей обойденных ядер,
образованных в результате  $\beta$-распада праматеринского  ядра,
индуцированного электромагнитным излучением. Это дает основание предполагать,
что такой процесс может быть определяющим при синтезе обойденных ядер в
звездном веществе.

\item[6.]

Показано, что при определенных условиях  заметный вклад в синтез
обойденных изотопов может также вносить $\beta$-распад из возбужденных
состояний праматеринского ядра.

\end{itemize}

Автор выражает глубокую благодарность И.В. Копытину за руководство работой;
Т.А. Чураковой и М.А. Долгополову --- за полезное обсуждение рассматриваемых
в диссертации вопросов.
%\pagebreak
%\listoffigures

%EOF
