\chapter{Столкновительный $\beta$-распад ядер в кулоновском поле.}

\subsection{Дифференциальное сечение  процесса столкновительного
$\beta$-распада ядер в кулоновском поле отталкивания.}

Рассмотрим нерелятивистское столкновение двух ядер $(A,Z)$ и $(A',Z')$,
первое из которых для процесса $\beta^-$-распада - материнское ядро, а второе
- столкновительный партнер (для простоты он будет считаться
бесструктурным).
Дифференциальное сечение процесса столкновительного $\beta$-распада
нуклида $(A,Z)$ можно  представить в виде:
\begin{eqnarray}\label{ds}
&&d\sigma^{(col)}_\beta={2\pi\mu\over{\hbar^2k_i}}\sum_{\beta_f}{\left| \bra{f} H^{(\beta^-)}
\ket{i}\right|}^2{d^3 K_f\over{(2\pi)^3}}{d^3 k_f\over{(2\pi)^3}}
{d^3 p_e\over{(2\pi\hbar)^3}}{d^3 p_\nu\over{(2\pi\hbar)^3}}\times\nonumber\\
&&\times\;\delta\left({\hbar^2 K_i^2\over{2 M}}+{\hbar^2 k_i^2\over{2\mu}}-{\hbar^2 K_f^2\over{2 M}}-
{\hbar^2 k_f^2\over{2\mu}}-E_e-E_\nu-\Delta-\Delta_f\right).
\end{eqnarray}
Здесь $\ket{i}$ и $\ket{f}$ - волновые функции начального и конечного состояний
столкновительной системы,
$\hbar \vec K_s$ и $\hbar \vec k_s$ - ее полный и относительный импульсы
в $s$-том состоянии ($s=i$ или $f$),
$\vec p_e$ и $\vec p_\nu$ - импульсы бета-электрона и антинейтрино,
$E_e$ и $E_\nu$ - их энергии,
$\Delta$ - пороговая энергия, определяемая разностью энергий связи
дочернего и материнского ядер (для $\beta$-стабильного ядра $\Delta>0$),
$\Delta_f$  - энергия состояния дочернего ядра, отсчитанная от
основного (см. рис.\ref{SH}),
$М$ и $\mu$ - полная и приведенная массы системы.
$\delta$-функция выражает закон сохранения энергии в процессе СБР.
Оператор ядерного бета-перехода  $H^{(\beta^-)}$ имеет вид \cite{aiz}:
\begin{eqnarray}\label{nH^beta}
H^{(\beta^-)}\approx {{1}\over{\sqrt{2}}}\sum_{j=1}^{A} \exp{( -i \vec q_\beta \vec r_j)}
(\tau_+)_j ( i g_v b_4 - g_a\, \vec b\, \vec \sigma_j ),
\end{eqnarray}
где $\vec r_j$ -- координата $j$-го нуклона;
$g_v$ и $g_a$ - векторная и псевдовекторная константы слабого
взаимодействия;
$\tau_+~=~(~\tau_1~+~i~\tau_2)/2$, $\;\tau_k,\sigma_k$ - операторы Паули;
$b_\lambda=i ( \bar u_e \gamma_\lambda \omega_\nu), u_e, \omega_\nu$ - лептонные
спиноры; $\gamma_\lambda (\lambda = 1,2,3,4 )$ - матрицы Дирака,
$\hbar \vec q_\beta=\vec p_e-\vec p_\nu$.

В (\ref{nH^beta}) опущены малые слагаемые, пропорциональные импульсам частиц,
участвующих в $\beta$-распадном процессе и предполагается, что пространственная
зависимость лептонных волновых функций имеет вид плоской волны.
При необходимости действие
кулоновского поля дочернего ядра на $\beta$-электрон можно учесть
впоследствии введением кулоновской функции
Ферми в конечное выражение для сечения процесса.

Переходя в систему центра масс, получим
\begin{eqnarray}\label{H^beta}
H^{(\beta^-)}=\exp{(-i\vec q_\beta\vec R_c)}\exp{( -i \vec \varkappa_\beta \vec R)}
\sum_{j=1}^A H^{(\beta^-)}_{j}\exp{(-i\vec q_\beta\vec\xi_{j})},
\end{eqnarray}
где  $\vec \varkappa_\beta=
{\mu\over Am}\vec q_\beta$, $m$ - масса нуклона,
$\vec R$ - относительная координата, $\vec R_c$~-~координата
центра тяжести системы,
$\xi_j$  - координата j-го нуклона, отсчитанная от центра тяжести материнского
ядра.

$H^{(\beta^-)}_{j}$ - $\beta$-распадный гамильтониан, действующий
на спиновые и изоспиновые координаты $j$-го нуклона \cite{aiz}:
\begin{eqnarray}
H^{(\beta^-)}_{j}\approx {1\over{\sqrt 2}}(\tau_+)_j ( i g_v b_4 - g_a\, \vec b\, \vec \sigma_j ).\nonumber
\end{eqnarray}


Представим волновые функции начального $(i)$ и конечного $(f)$ состояний
столкновительной ядро-ядерной системы в виде:
\begin{eqnarray}
\ket{s}= \Psi(\vec k_s, \vec R)\exp(i \vec K_s \vec R_c) \ket{ \beta_s}.
\end{eqnarray}
Здесь
$\ket{ \beta_s}$ - волновые функции, характеризующие внутренние состояния
материнского ($s=i$) или дочернего ($s=f$) ядер,
$\Psi (\vec k_s, \vec R)$ - волновые функции относительного движения в
столкновительной системе (ядра $(A,Z)$, $(A',Z')$ при $s=i$
и $(A,Z+1)$, $(A',Z')$ при $s=f$).

Рассмотрим вначале задачу рассеяния в кулоновском поле отталкивания, где имеется  спектр только положительных собственных
значений энергии. Как известно \cite{landau}, в этом случае правильное асимптотическое поведение волновой функции
получается, если для начального состояния она представляет собой суперпозицию плоской и расходящейся, а для конечного
состояния - плоской и сходящейся сферической волн. Выражения для соответствующих кулоновских волновых функций известны
\cite{landau}:
\begin{eqnarray}\label{psi_i}
\Psi (\vec k_i, \vec R)=e^{-{\pi \lambda_i\over2}} \Gamma(1+i\lambda_i)
e^{i \vec k_i \vec R} {\rm F} \left(- i \lambda_i,1;i(k_i R - \vec k_i \vec R)\right),
\end{eqnarray}

\begin{eqnarray}\label{psi_f}
\Psi (\vec k_f, \vec R)=e^{-{\pi \lambda_f\over2}} \Gamma(1-i\lambda_f)
e^{i \vec k_f \vec R} {\rm F} \left(i \lambda_f,1;-i(k_f R + \vec k_f \vec R)\right).
\end{eqnarray}
Здесь
\begin{eqnarray}
\lambda_i = {Z Z' e^2 \mu\over \hbar^2 k_i},
\lambda_f = {(Z+1) Z' e^2 \mu\over \hbar^2 k_f}.
\end{eqnarray}
С учетом (\ref{psi_i}), (\ref{psi_f}) матричный элемент процесса СБР принимает вид:
\begin{multline}\label{pred_mat}
\bra{f} H^{(\beta^-)}\ket{i}= \bra{\beta_f }
\sum_{j=1}^A H^{(\beta^-)}_{j}\exp{(-i\vec q_\beta\vec\xi_{j})}
\ket{ \beta_i}\times \\
\times(2\pi)^3 \delta (\vec K_i - \vec K_f - \vec q_\beta)
e^{-{\pi\over 2} (\lambda_i + \lambda_f)} \Gamma{(1+i\lambda_i)}
\Gamma{(1+i\lambda_f)}\times\\
\times\int e^{i(\vec k_i - \vec k_f - \vec \varkappa_\beta)\vec R}
{\rm F} \left(-i \lambda_i,1;i(k_i R - \vec k_i \vec R)\right)\times\\
\times{\rm F} \left(-i \lambda_f,1;i(k_f R + \vec k_f \vec R)\right) d^3 R .\qquad
\end{multline}

Для вычисления интеграла типа:
\begin{equation}\label{def_J}
J(\lambda)\equiv\int d^3 r\, e^{-\lambda r} {e^{i\vec q \vec r}\over r}
{\rm F} (i a_1,1;i(p_1 r - \vec p_1 \vec r))
{\rm F} (i a_2,1;i(p_2 r + \vec p_2 \vec r))
\end{equation}
можно воспользоваться способом, в котором каждая из вырожденных гипергеометрических функций
заменяется выражением в виде контурного интеграла \cite{nord, bess}. В результате
получается:
\begin{equation}\label{rez_J}
J(\lambda)={{2\pi}\over\alpha} e^{-\pi a_1}\left({\alpha\over\gamma}\right)^{i a_1}
\left({{\gamma+\delta}\over\gamma}\right)^{-i a_2}{\rm F}\left(1-i a_1, i a_2,
1;{{\alpha\delta-\beta\gamma}\over{\alpha(\gamma+\delta)}}\right),
\end{equation}
где
\begin{equation*}
\begin{split}
&\alpha={1\over2}(q^2+\lambda^2),\qquad \beta=\vec p_2 \vec q - i \lambda p_2,
\\
&\gamma=\vec p_1 \vec q + i \lambda p_1 - \alpha, \quad \delta=p_1 p_2+
\vec p_1 \vec p_2 -\beta.
\end{split}
\end{equation*}

Из сравнения (\ref{pred_mat}) и (\ref{def_J}) видно, что соответствующий интеграл,
необходимый для вычисления матричного  элемента процесса СБР, может
быть найден на основе формулы (\ref{rez_J}), если положить:
\begin{multline*}
\int e^{i(\vec k_i - \vec k_f -  \vec \varkappa_\beta)\vec R}
{\rm F} \left(-i \lambda_i,1;i(k_i R - \vec k_i \vec R)\right)\times\\
\times{\rm F} \left(-i \lambda_f,1;i(k_f R + \vec k_f \vec R)\right) d^3 R=
\left.-{{\partial J}\over{\partial \lambda}}\right|_{\lambda=0}
\end{multline*}
и установить следующую связь между параметрами:
$$
a_1=-\lambda_i, \qquad a_2=-\lambda_f, \qquad \vec q= \vec k_i - \vec k_f - \vec \varkappa_\beta,
$$
$$
\vec p_1 =\vec k_i,  \qquad \vec p_2 =\vec k_f.
$$
В результате получаем:
\begin{multline}\label{mat^2}
\mod{\bra{f} H^{(\beta^-)}\ket{i}}^2
=4\pi^2b^{-2}\bigg| \bra{\beta_f } \sum_{j=1}^A H^{(\beta^-)}_{j}\exp{(-i\vec q_\beta\vec\xi_{j})}
\ket{ \beta_i}\bigg|^2
\times\\
\times
\left[(2\pi)^3 \delta (\vec K_i - \vec K_f - \vec q_\beta)\right]^2
e^{{\pi} (\lambda_i - \lambda_f)}{{\mod{\Gamma{(1+i\lambda_i)}}^2
\mod{\Gamma{(1+i\lambda_f)}}^2}} \\
\times \left| A
{\rm F} (1+i \lambda_i,-i \lambda_f,1;\zeta)
+B
(1+i \lambda_i)\lambda_f {\rm F} (2+i \lambda_i,1-i \lambda_f,2;\zeta)
 \right|^2.
\end{multline}
Здесь
\begin{eqnarray}\label{zeta}
\zeta={{b f - c d}\over{ab}},
\end{eqnarray}
и введены обозначения:
\begin{equation}\label{AB}
\begin{split}
&A \equiv {{\lambda_f(k_i+k_f)}\over{a}}-{{k_i(\lambda_f-\lambda_i)}\over{c}},\\
&B \equiv {{k_i d-k_f g}\over{ab}}-{{(k_i+k_f)(b f-c d)}\over{a^2 b}}.
\end{split}
\end{equation}
Кроме того,
\begin{eqnarray}
a={1\over2}(\vec k_i^2+\vec k_f^2-\vec \varkappa_\beta^2)+k_i k_f,
\quad b={1\over2}(\vec k_i^2+\vec k_f^2+\vec \varkappa_\beta^2)-\vec k_i \vec k_f
-\vec \varkappa_\beta\vec k_i+\vec \varkappa_\beta\vec k_f,\nonumber\\
c={1\over2}(\vec k_i^2-\vec k_f^2-\vec \varkappa_\beta^2)- \vec \varkappa_\beta \vec k_f,
\quad d=\vec k_i \vec k_f-k_f^2-\vec \varkappa_\beta\vec k_f,\nonumber\\
g=k_i^2-\vec k_i \vec k_f-\vec \varkappa_\beta\vec k_i,
\quad f= k_i k_f+k_f^2+\vec \varkappa_\beta\vec k_f\nonumber.
\end{eqnarray}

Пусть в дочернем ядре имеются состояния,
допускающие $\beta$-переходы разрешенного типа. Учитывая их как
наиболее интенсивные, можно положить $e^{-i\vec q_\beta\vec\xi_{j}}{\approx}1$.
Тогда для ядерного матричного элемента  $\beta$ -перехода
$\ket{\beta_i}\to\ket{\beta_f}$ получим:
\begin{multline}\label{mat_nuc}
\bigg|\bra{\beta_f }  \sum_{j=1}^A H^{(\beta^-)}_{j}
e^{-i\vec q_\beta\vec\xi_{j}}\ket{ \beta_i}\bigg|^2\approx
g^2_v\left\{\mod{M_v}^2+(g_a/g_v)^2\mod{M_a}^2\right\}\equiv\\
\equiv
g^2_v \xi_\beta(\beta_f),
\end{multline}
где  $M_v$ и $M_a$ - соответствующие ядерные матричные элементы
для $\beta$-перехода разрешенного типа:
\begin{eqnarray}
M_v=\bra{\beta_f } \sum_{j=1}^A \tau_+^{(j)}) \ket{ \beta_i},\qquad
M_a=\bra{\beta_f } \sum_{j=1}^A \vec\sigma^{(j)}\tau_+^{(j)}) \ket{ \beta_i}.\nonumber
\end{eqnarray}
В реальной ситуации за редкими исключениями отличен от нуля только
гамов-теллеров\-cкий матричный элемент $M_a$.

Выбирая в качестве оси z направление начального импульса относительного движения
ядер $(\vec k_i )$ и полагая $\vec K_i =0 $,  можно  с учетом (\ref{mat^2})
частично  проинтегрировать (\ref{ds}):
\begin{eqnarray}\label{rez_ds}
&&d\sigma^{(col)}_\beta=
(2 \pi c \hbar^3)^{-4}
g_v^2 \alpha_e^2 Z (Z+1) {Z'}^2 \mu^4
\sum_{\beta_f}\xi_\beta(\beta_f)\times\nonumber\\
&&\times {{\big| A {\rm F} (1+i \lambda_i,-i \lambda_f,1;\zeta) + B (1+i \lambda_i)\lambda_f
{\rm F} (2+i \lambda_i,1-i \lambda_f,2;\zeta)\big|^2}\over{ k_i^2 b^2
(1-\exp(-2\pi\lambda_i))(\exp(2\pi\lambda_f)-1)}}\times\nonumber\\
&&\times {(E_e^2 - m_e^2 c^4)}^{1/2}{(\varepsilon_i - \varepsilon_f - E_e - \Delta
- \Delta_f)}^2  E_e \;dE_e \; d\Omega_e \; d\Omega_{\nu}\;d\varepsilon_f \; d\Omega_f \equiv\nonumber\\
&&\equiv \sum_{\beta_f} d\sigma^{(col)}_\beta (\beta_f).
\end{eqnarray}
Здесь $\Omega_f\equiv (\theta_f,\phi_f)$, $\Omega_e\equiv (\theta_e,\phi_e)$,
$\Omega_\nu\equiv (\theta_\nu,\phi_\nu)$ - углы, задающие направление
векторов ${\vec k}_f$, ${\vec k}_e$ и ${\vec k}_\nu$ соответственно,
$\alpha_e$ - постоянная тонкой
структуры, $\varepsilon_s$ - энергия относительного движения в столкновительной
системе, $m_e$ - масса электрона. При получении (\ref{rez_ds}) также учтено, что
\begin{eqnarray}
&&\mod{\Gamma{(1+i\lambda_i)}}^2 e^{\pi\lambda_i} = {{\pi\lambda_i e^{\pi\lambda_i}}
\over{\sh (\pi\lambda_i)}} = {{2 \pi\lambda_i}\over{1- \exp{(-2\pi\lambda_i)}}},\nonumber\\
&&\mod{\Gamma{(1+i\lambda_f)}}^2 e^{- \pi\lambda_f} = {{2 \pi\lambda_f}\over{\exp{(2 \pi\lambda_f)-1}}},
\end{eqnarray}


а импульс антинейтрино
$p_\nu =(\varepsilon_i - \varepsilon_f - E_e - \Delta - \Delta_f)/{c}$.

%%%%%%%%%%%%%%%%%%%%%%%%%%%%%%%%%%%%%%%%%%%%%%%%%%%%%%%%%%%%%%%%%%%%%%%%%%%%
%%%%%%%%%%%%%%%%%%%%%%%%%%%%%%%%%%%%%%%%%%%%%%%%%%%%%%%%%%%%%%%%%%%%%%%%%%%%%


\section{Полное сечение процесса СБР. Результаты расчетов.}

Расчет полного сечения процесса СБР на основе выражения (\ref{rez_ds}) довольно затруднителен из-за необходимости
интегрирования по направлениям вылета лептонов. Однако, задачу можно упростить без существенной потери точности, если
воспользоваться аналогией с электромагнитными переходами в кулоновском поле между состояниями непрерывного спектра
(задача тормозного излучения). Как известно \cite{landau}, в этом случае для применения дипольного приближения достаточно
нерелятивистских скоростей у сталкивающихся частиц. В нашем случае мы также рассматриваем ядро-ядерные столкновения в
кулоновском поле при нерелятивистских скоростях, так что это условие тоже имеет место. Кроме того, в диапазоне
столкновительных энергий, характерных для процесса СБР,  практически
$ \varkappa_\beta \lesssim 0,1 \;\rm {Фм^{-1}}$.
Все это вместе позволяет в (\ref{rez_ds}) положить
$\varkappa_\beta\approx 0$, что эквивалентно ``дипольному'' приближению
$exp{(- i \vec \varkappa_\beta \vec R)} \approx 1$ в формуле (\ref{H^beta}).
Тогда (\ref{rez_ds}) существенно упрощается и можно проинтегрировать
по энергии и направлениям
вылета $\beta$-электрона и антинейтрино. В результате получим (здесь и дальше используем
систему единиц  $m_e = \hbar = c = 1$):
\begin{multline} \label{dsech}
d\sigma^{(col)}_\beta(\beta_f)=
{32\sqrt{2}\over \pi}  {{g^2_v\alpha_e^2{Z'}^2 \mu^{9/2}
\lambda_i} \over k_i}\xi_\beta(\beta_f)
\times\\
\times
{{\Phi (E_f) |{\rm F}(1+i \lambda_i, -i\lambda_f,1;\zeta)|^2}\over
{(1-\exp(-2\pi\lambda_i))(\exp(2\pi\lambda_f)-1))}}
{{\lambda_f \varepsilon_f^{1/2}d\varepsilon_f\sin \theta_f
d\theta_f}\over{(\vec k_i-\vec k_f)^4(k^2_i-k^2_f)^2}}.
\end{multline}
$E_f=\varepsilon_i-\varepsilon_f-\Delta-\Delta_f$,
а функция $\Phi (E)$ имеет вид
$$
\Phi (E)={1\over 60}(E^2-1)^{1/2}(2E^4-9E^2-8)+{1\over 4}E\ln{(E+(E^2-1)^{1/2})}.
$$
Параметр $\zeta$ (см.~(\ref{zeta})) теперь равен:
\begin{eqnarray}
\zeta={{2(1-\cos\theta_f)}\over{k_i/k_f+k_f/k_i-2\cos\theta_f}}\quad.\nonumber
\end{eqnarray}
С учетом известного преобразования гипергеометрической функции:
\begin{eqnarray}
{\rm F}(a,b,c;x)=(1-x)^{-a}{\rm F}\left(a,c-b,c;{x\over{x-1}}\right),\nonumber
\end{eqnarray}
сечение столкновительного $\beta$-распада  приводится к виду:
\begin{multline} \label{sech}
 \sigma^{(col)}_\beta(\beta_f)=
{4\sqrt{2}\over \pi} {{g^2_v\alpha_e^4Z (Z+1) {Z'}^4 \mu^{9/2}}
\over{\varepsilon_i^{3/2} (1-exp(-2\pi\lambda_i))}}
\xi_\beta(\beta_f)
\times\\
\times\int\limits_0^{\varepsilon_i{-}\Delta{-}\Delta_f}\!\!\!\!\!
{{\Phi(E_f) d\varepsilon_f}\over{(exp(2\pi\lambda_f)-1)
k_f ( k_i- k_f)^4(k_i+k_f)^2}}\times\\
\times\int\limits_{x_0}^0 {{|{\rm F}(-i \lambda_i, -i\lambda_f,1; x)|^2}\over{(1-x)^2}} dx,
\end{multline}
где  $x_0=-4 k_i k_f/(k_i- k_f)^2$.

%Вставка "после формулы (17)"

Результаты расчетов величины    $\sigma^{(col)}_\beta(\beta_f)$
по формуле (\ref{sech}) как функции начальной энергии относительного движения
$\varepsilon_i$ и для различных значений $\Delta$  представлены на рис.\ref{MO961}, \ref{MO962}
(рассматривались столкновения ядра $^{96} Mo$ с протоном, $\alpha$-частицей
и ядрами $^7 Li$ и $^{16} O$).

\begin{figure}
\vspace{18 true cm}
\caption{{Зависимость сечения  $\sigma^{(col)}_\beta$ от энергии относительного
движения ядер $\varepsilon_i$ в единицах $\xi_\beta$ при столкновении
ядра ${}^{96}Mo$ с протоном (1), $\alpha$-частицей (2) и ядрами ${}^7 Li$ (3), ${}^{16} O$(4):
$\Delta=1 МэВ$.}}
\label{MO961}
\end{figure}

\begin{figure}
\vspace{18 true cm}
\caption{{ Зависимость сечения  $\sigma^{(col)}_\beta$ от энергии относительного
движения ядер $\varepsilon_i$ в единицах $\xi_\beta$ при столкновении
ядра ${}^{96}Mo$ с протоном (1), $\alpha$-частицей (2) и ядрами ${}^7 Li$ (3), ${}^{16} O$(4):
$\Delta=3 МэВ$.}}
\label{MO962}
\end{figure}


Как видно из рис. \ref{MO961}, \ref{MO962}, сечение процесса СБР сильно зависит от зарядового числа
$Z'$ столкновительного партнера. Несмотря на  наличие фактора ${Z'}^4$
в формуле (\ref{sech}),
неявная зависимость от  $Z'$ такова, что с его ростом
величина полного сечения
резко уменьшается   при  значениях $\varepsilon_i$,
сравнимых по величине с пороговой энергией $\Delta$.
Такой результат не является неожиданным, поскольку известен в теории тормозного
излучения заряженных частиц в кулоновском поле. Из-за сильной (экспоненциальной)
зависимости кулоновских волновых функций от зарядового числа
величина сечения процесса эмиссии тормозных $\gamma$-квантов резко уменьшается
при замене
притяжения на отталкивание и быстро стремится к нулю с ростом зарядовых чисел.
Для  процесса СБР ядра, по сути дела, мы используем тот же
физический механизм, только, в отличие от случая тормозного излучения, электромагнитая
вершина заменена на слабую.

\begin{figure}
\vspace{18 true cm}
\caption{{ Зависимость сечения  $\sigma^{(col)}_\beta$ от энергии относительного
движения ядер $\varepsilon_i$ и пороговой энергии $\Delta$ в единицах $\xi_\beta$ при столкновении
ядра ${}^{96}Mo$ с $\alpha$-частицей (а) и протоном (б):
$\Delta=1 МэВ$ (1) и $\Delta=3 МэВ$ (2).}}
\label{MO96P}
\end{figure}




Как видно из  рис.\ref{MO96P}, где представлена зависимость
сечения  $\sigma^{(col)}_\beta$ от энергии относительного
движения ядер $\varepsilon_i$   при столкновении
ядра ${}^{96}Mo$ с $\alpha$-частицей  и протоном при разных значениях  пороговой энергии
$\Delta=1 МэВ$  и $\Delta=3 МэВ$,
уменьшение $\Delta$ увеличивает выход реакции
СБР, однако, не меняет указанной чувствительности сечения  к зарядовому числу
столкновительного партнера.


%%%%%%%%%%%%%%%%%%%%%%%%%%%%%%%%%%%%%%%%%%%%%%%%%%%%%%%%%%%%%%%%%%%%%%%%%%%%%%%%%%%

%%%%%%%%%%%%%%%%%%%%%%%%%%%%%%%%%%%%%%%%%%%%%%%%%%%%%%%%%%%%%%%%%%%%%%%%%%%%%%%%%%%%%

\section{Сечение столкновительного $\beta$-распада\\ ядер в
борновском приближении.}

Рассмотрим теперь предельные случаи формулы (\ref{sech}).
Пусть $\lambda_i \ll 1,  \lambda_f \ll 1.$
Это есть условие применимости борновского приближения в
кулоновском поле.
Тогда
\begin{eqnarray}
{\rm F}(-i\lambda_i,-i\lambda_f,1;x)\approx {\rm F}(0,0,1;x)=1.\nonumber
\end{eqnarray}

В этом случае для дифференциального сечения процесса СБР получим:
\begin{eqnarray}\label{born}
&&d\sigma^{(col)}_\beta(\beta_f)\approx
{{g_v^2\alpha_e^2  {Z'}^2}\over{\pi^3\varepsilon^{1/2}_i}}{\xi_\beta(\beta_f)
\Phi(E_f)\varepsilon_f^{1/2}\,d\varepsilon_f\over (\varepsilon_i-\varepsilon_f)^4}.
\end{eqnarray}
Это есть сечение столкновительного бета-распада в борновском приближении. Видно, что оно  растет пропорционально квадрату
зарядового числа столкновительного партнера и через фактор $\Phi(E_f)$ зависит от соотношения пороговой и
столкновительной энергий. Формула (\ref{born}) полностью соответствует выражению, полученному ранее во втором порядке по
теории возмущений \cite{batkin} (соответствующая диаграмма представлена на рис.~\ref{DBORN}, где $V_c$ -- потенциальная
энергия кулоновского взаимодействия ядер).

\begin{figure}
\vspace{5 true cm}
\caption{Диаграмма процесса столкновительного \be-распада, рассматриваемого
в приближении плоских волн.}
\label{DBORN}
\end{figure}


На рис. \ref{RBORN} приведены графики зависимости величины $\sigma^{(col)}_\beta$,
рассчитанные по формулам (\ref{sech}) и (\ref{born}) (в обоих случаях
рассматривались столкновения ядра $^{80}Se$ с протоном и ядрами
$^{28}Si$ и $^{56}Fe$). Сравнение представленных результатов показывает,
что в области столкновительных энергий $\varepsilon_i\gtrsim\Delta$,
наблюдается сильное отличие величин полного сечения, рассчитанных
с точным учетом
кулоновского взаимодействия и в борновском приближении. С ростом
зарядового числа столкновительного партнера эти расхождения увеличиваются.


\begin{figure}
\vspace{18 true cm}
\caption{{ Зависимость сечения  $\sigma^{(col)}_\beta$ от энергии относительного
движения ядер $\varepsilon_i$ (пороговая энергия $\Delta=1.87 \; МэВ$)  при столкновении
ядра ${}^{80}Se$ с протоном (а) и ядрами $^{28}Si$ (б), $^{56}Fe$(в).
Тонкая линия -- с точным учетом кулоновского взаимодействия,
жирная -- в борновском приближении.}}
\label{RBORN}
\end{figure}



%%%%%%%%%%%%%%%%%%%%%%%%%%%%%%%%%%%%%%%%%%%%%%%%%%%%%%%%%%%%%%%%%%%%%%%%%%%%%%
%%%%%%%%%%%%%%%%%%%%%%%%%%%%%%%%%%%%%%%%%%%%%%%%%%%%%%%%%%%%%%%%%%%%%%%%%%%%%%%%%%%%

\section{Предельный случай больших зарядовых чисел столкновительного партнера
при малых и промежуточных \\ энергиях столкновения.}

Представляет интерес и другой предельный случай, когда
$  \lambda_i \gg 1,  \lambda_f \gg 1.$
Большие значения параметров позволяют для гипергеометрической функции  использовать предельный переход \cite{zomm}:
\begin{eqnarray}\label{perehod}
|{\rm F}(-i\lambda_i,-i\lambda_f,1;x)|^2
{\longrightarrow}f(\lambda_i,\lambda_f,\theta_f) \exp{(2 \pi \lambda_i)},
\end{eqnarray}
где  $f(\lambda_i,\lambda_f,\theta_f)$ - медленно меняющаяся функция.
С учетом того, что выражение
$$
\exp\left({{i \pi (p+1)}/{2}}\right){\rm H}_{p}(i u)
$$
( ${\rm H}_{p}(i u)$ - функция Ханкеля )
вещественно при вещественных значениях u, ее можно представить следующим образом:
\begin{eqnarray}
f(\lambda_i,\lambda_f,\theta_f)&=&{1\over{192}}\left\{ (\pi - \theta_f)^2
e^{{5 i \pi}\over{6}}{\rm H}^{(2)}_{2/3}(i u) -
2(\pi - \theta_f) e^{{2 i \pi}\over{3}}{\rm H}^{(1)}_{2/3}(i u) \right\}^2 ,\nonumber
\end{eqnarray}
\begin{eqnarray}
u={{(\pi - \theta_f)^3}\over 6}{{\lambda_i \lambda_f ( \lambda_f - \lambda_i )}
\over{( \lambda_f + \lambda_i )}}.\nonumber
\end{eqnarray}
Преобразовав (\ref{dsech}) с учетом (\ref{perehod}), получим:
\begin{multline} \label{dsechbig}
d\sigma^{(col)}_\beta(\beta_f)=
{32\sqrt{2}\over \pi}{{g^2_v\alpha_e^2{Z'}^2
\mu^{9/2}\lambda_i} \over k_i}{{\xi_\beta(\beta_f)
\lambda_f \varepsilon_f^{1/2}d\varepsilon_f\sin \theta_f
d\theta_f}\over{(\vec k_i-\vec k_f)^4(k^2_i-k^2_f)^2}}
\times\\
\times
{{\Phi (E_f) f(\lambda_i, \lambda_f,\theta_f)}\over
{(1-\exp(-2\pi\lambda_i))(\exp(2\pi(\lambda_f-\lambda_i))-\exp{(-2\pi\lambda_i)})}}
\end{multline}
Видно, что сечение СБР экспоненциально зависит от параметра
$\lambda=\lambda_f -\lambda_i$ .
Так как $\varepsilon_i >\varepsilon_f$, то $0<\lambda <\lambda_f$,
и при больших $\lambda$  вероятность
столкновительного бета-распада ничтожно мала.
В явном виде:
\begin{eqnarray} \label{lambda}
\lambda= \alpha_e Z' \sqrt{{{\mu}\over{2}}}\left({{Z+1}\over
{\sqrt{\varepsilon_f}}}-{{Z}\over{\sqrt{\varepsilon_i}}}\right)
\end{eqnarray}
и оно будет тем меньше (сечение соответственно тем больше), чем легче
сталкивающиеся ядра, больше относительная энергия столкновения и
меньше пороговая энергия $\Delta$~ (~ в~ последнем~ случае при $\Delta\to 0$
$\;\varepsilon_f\to\varepsilon_i$ ). Этот результат фактически проиллюстрирован
на рис. \ref{MO961}, \ref{MO962}, поскольку в рассматриваемом диапазоне энергий
и для указанных
значений зарядовых чисел имеет место условие $\lambda_i,\,\lambda_f\gg 1$.


%%%%%%%%%%%%%%%%%%%%%%%%%%%%%%%%%%%%%%%%%%%%%%%%%%%%%%%%%%%%%%%%%%%%%%%%%%%%
%%%%%%%%%%%%%%%%%%%%%%%%%%%%%%%%%%%%%%%%%%%%%%%%%%%%%%%%%%%%%%%%%%%%%%%%%%%%%%%%


\section{Сечение столкновительного $\beta$-распада в кулоновском поле
притяжения.}

Все полученные формулы относятся к кулоновскому полю отталкивания. Сечение
СБР в поле притяжения получается из (\ref{dsech}), (\ref{sech}) заменой:
$$
\lambda_i{\rightarrow} {-} \lambda_i,\qquad  \lambda_f{\rightarrow} {-} \lambda_f.
$$
При этом формула (\ref{born}), полученная в борновском приближении, не меняется,
а из формулы (16) в случае
$ \;~ \lambda_i,~ \lambda_f~ \gg~ 1~,$ получим:
\begin{multline}\label{dsechot}
d \sigma^{(col)}_\beta(\beta_f)=
{16  \over \pi}{{g^2_v \alpha_e^4Z (Z+1){Z'}^4\mu^5}
\over \varepsilon_i}{\xi_\beta(\beta_f) f(\lambda_i,\lambda_f,\theta_f)
\Phi(E_f)
\over{(\vec k_i-\vec k_f)^4(k^2_i-k^2_f)^2}}
\times\\
\times
{{d\varepsilon_f\sin \theta_f d\theta_f}\over{(1-\exp(-2\pi\lambda_f))
(1-\exp(-2\pi\lambda_i))}}.
\end{multline}
Как следует из (\ref{dsechot}), сечение растет пропорционально $Z (Z+1){Z'}^4$ и нет
экспоненциально затухающего множителя:  когда
$ \; \lambda_i,\; \lambda_f \gg 1,$ фактор
$(1-\exp(-2\pi\lambda_f))(1-\exp(-2\pi\lambda_i))\rightarrow~ 1$.

Поведение сечения столкновительного $\beta$-распада в кулоновском поле
притяжения как функции энергии столкновения и зарядовых чисел ядер-партнеров
существенно меняется по сравнению с его поведением в поле отталкивания
(\ref{dsechbig}), так как
в последнем есть экспоненциальная зависимость вида
$\exp( - 2\pi \lambda)$ ( $\lambda $ определено в (\ref{lambda}), которая
подавляет аналогичную формуле (\ref{dsechot}) степенную зависимость
от зарядовых чисел сталкивающихся ядер. С ростом $Z$ и $Z'$ этот эффект
еще больше усиливается, что находит отражение в сильном уменьшении
соответствующих значений сечения СБР  в поле отталкивания.

Это проиллюстрировано на рис. \ref{MODY},\ref{CDOS} , где для процесса СБР ядер $^{96}Mo$ и $^{164}Dy$
при пороговой энергии $\Delta=1\; МэВ$, и  $^{114}Cd$ и $^{190}Os$ при пороговой энергии $\Delta=2\; МэВ$
показана зависимость полного сечения от энергии при их столкновениях с электронами
и протонами. Видно, что при $\varepsilon_i<10\,Мэв$ величины сечений
отличаются на несколько порядков в зависимости от знака заряда столкновительного
партнера, и это при существенном различии в величине приведенной массы $\mu$
(при столкновении с электроном $\mu\approx m_e$, а с протоном~-~$\mu\approx m$).

\begin{figure}
\vspace{18 true cm}
\caption{{ Зависимость сечения  $\sigma^{(col)}_\beta$ от энергии относительного
движения ядер $\varepsilon_i$ (пороговая энергия $\Delta=1\; МэВ$)  при столкновении
ядра ${}^{96}Mo$ с  протоном (1) и электроном (1') (сплошные линии) и $^{164}Dy$ с  протоном (2) и электроном (2')(пунктирные линии).
}}
\label{MODY}
\end{figure}

\begin{figure}
\vspace{18 true cm}
\caption{{ Зависимость сечения  $\sigma^{(col)}_\beta$ от энергии относительного
движения ядер $\varepsilon_i$ (пороговая энергия $\Delta=2\; МэВ$)  при столкновении
ядра ${}^{114}Cd$ с  протоном (1) и электроном (1') (сплошные линии) и $^{190}Os$ с  протоном (2) и электроном (2')(пунктирные линии).
}}
\label{CDOS}
\end{figure}


Такой "зарядный" эффект, который даже важнее эффекта масс, наблюдается также при рассмотрении явления тормозного
излучения \cite{zomm},
\cite{ahi}, \cite{BLP}.

Сформулируем основные результаты, полученные в данной главе.


Построена теория
процесса столкновительного $\beta$-распада стабильного ядра, инициированного
ядро-ядерными кулоновскими столкновениями.

Расчеты полных сечений этого процесса, выполненные с точными кулоновскими
функциями, показали, что
в области столкновительных энергий, сравнимых по величине с пороговой
энергией $\Delta$,
приближение плоских волн дает сильно завышенные результаты. Эти расхождения
тем заметнее, чем больше зарядовое число столкновительного партнера.

В случае, когда зарядовое число столкновительного партнера $Z'\gg 1$, выход
продуктов реакции столкновительного $\beta$-распада, инициированного
ядро-ядерными столкновениями, можно считать пренебрежимо малым.

Выявлена чувствительность полных сечений процесса столкновительного
$\beta$-распада, стимулированного кулоновскими ядро-ядерными столкновениями
к величине $\Delta$
при малых энергиях относительного движения $\varepsilon_i$.
При $\varepsilon_i \gg \Delta$, как и следовало ожидать, различие между
величинами сечений становится незначительным.

Основные результаты, полученные в данной главе, опубликованы в работах
\cite{K5, K6, K7}.
%EOF
