\chapter{Столкновительный $\beta$-распад стабильных ядер, стимулированный
        нейтронами.}

В первой главе рассматривались только кулоновские столкновения: протон-ядерные и
ядро-ядерные. Рассчитанные сечения оказались невелики (менее
$10^{-50} см^2$), и процесс пока не доступен для прямого наблюдения.
Тем не менее существует возможность получения косвенных свидетельств в пользу
его реальности. Как оказалось, столкновительный $\beta$-распад стабильных ядер,
инициируемый их кулоновскими столкновениями с другими ядерными частицами
звездной среды, может быть основой модели процесса синтеза обойденных ядер.
Как указывалось во Введении, проблема их синтеза на основе физического механизма
захвата нейтронов ($s$- или $r$-процесса)  состоит в прерывании
цепочки последовательных $\beta^-$-распадов на $\beta$-стабильном ядре $(A,Z)$.
Его \be-стабильность обусловлена энергетическим барьером, запрещающим
естественный \be-переход $(A,Z)\to(A,Z+1)$ и, соответственно, появление
по этому каналу обойденного стабильного изобара $(A,Z+2)$.
Механизм столкновительного $\beta$-распада позволяет преодолеть указанный энергетический
барьер за счет столкновительной энергии и осуществить по этому каналу $\beta^-$-переход
$(A,Z) \to (A,Z+1)$, после чего становиться возможным и естественный $\beta^-$-распад
 $(A,Z+1) \to (A,Z+2)$. Может оказаться, что при этом малость сечений
 для процесса такого рода уже не будет играть особой роли, если будут не малы плотность
 вещества в недрах звезды и временная протяженность квазиравновесной стадии
 звездной эволюции.
 Изучению этого вопроса далее будет посвящена глава 3.
 Здесь же мы отметим, что наряду с кулоновскими столкновениями ядер можно предложить
 механизм СБР, не связанный с кулоновскими силами и в то же время незамаскированный
 возможным появлением продуктов \be-распада за счет ядерных реакций.
 Речь идет о процессе СБР ядра, стимулированном столкновениями с нейтронами.
 В этом случае
величина сечения процесса столкновительного $\beta$-распада может стать существенно больше,
так как вместо
кулоновского взаимодействия будет иметь место сильное. Однако при температурах  $T\lesssim 500 кэВ$,
когда идет синтез ядер в недрах звезд, кулоновский барьер не позволяет
ядерным частицам сблизится на расстояние действия ядерных сил. Единственно,
для кого кулоновский барьер будет не существенен -- это нейтроны. Известно, что
в звездной среде они всегда есть в довольно большом количестве.
Столкновения нейтронов  со стабильными ядрами $(A,Z)$ также
могут стимулировать $\beta^-$-распад по механизму СБР, но, в отличие от кулоновских,
ответственным за этот распад будет уже сильное взаимодействие. Столкновительный
$\beta$-распад стабильных ядер, инициированный нейтронами звездной среды,
будет дополнять физический механизм образования обойденных ядер за счет
кулоновских столкновений. Конечно, практическая значимость этого процесса будет
определяться сравнительными величинами кулоновского и ядерного сечений процесса СБР,
с одной стороны, и соотношением плотностей легких ядер и нейтронов в среде, с другой.


\section{Сечение процесса.}

Рассмотрим процесс столкновительного $\beta^-$-распада ядра $(A,Z)$ при
столкновении его с нейтроном.

Если, как и в главе 1, обозначить через $\hbar \vec K_s$ и $\hbar \vec k_s$ соответственно
полный  и относительный импульсы системы (ядро $(A,Z)$-нейтрон) в начальном
($s=i$) и системы (ядро $(A,Z+1)$-нейтрон) в конечном ($s=f$) состояниях,
то для дифференциального сечения процесса СБР стабильного ядра $(A,Z)$
можно воспользоваться общим выражением (\ref{ds}) (остальные обозначения
входящих в него величин сохраняют тот же смысл). Как и раньше, для оператора
ядерного \be-перехода $H^{\beta^-}$, определенного в с.ц.и. системы
(ядро $(A,Z)$-нейтрон), можно воспользоваться выражением (\ref{H^beta})


%Исходя из общей формулы для дифференциального сечения процесса столкновительного
%$\beta$-распада нуклида $(A,Z)$ (\ref{ds}):

%\begin{eqnarray*}
%&&d\sigma^{(col)}_\beta={2\pi\mu\over{\hbar^2k_i}}\sum_{\beta_f}{\left| \bra{f} H^{(\beta^-)}
%\ket{i}\right|}^2{d^3 K_f\over{(2\pi)^3}}{d^3 k_f\over{(2\pi)^3}}
%{d^3 p_e\over{(2\pi\hbar)^3}}{d^3 p_\nu\over{(2\pi\hbar)^3}}\times\nonumber\\
%&&\times\;\delta\left({\hbar^2 K_i^2\over{2 M}}+{\hbar^2 k_i^2\over{2\mu}}-{\hbar^2 K_f^2\over{2 M}}-
%{\hbar^2 k_f^2\over{2\mu}}-E_e-E_\nu-\Delta-\Delta_f\right),
%\end{eqnarray*}
%где, как и раньше,
%$\hbar \vec K_s$ и $\hbar \vec k_s$ - ее полный и относительный импульсы
%в s-том состоянии ($s=i$ или $f$),
%$\vec p_e$ и $\vec p_\nu$ - импульсы бета-электрона и антинейтрино,
%$E_e$ и $E_\nu$ - их энергии,
%$\Delta$ - пороговая энергия, определяемая разностью энергий связи
%дочернего и материнского ядер (для $\beta$-стабильного ядра $\Delta>0$),
%$\Delta_f$  - энергия состояния дочернего ядра, отсчитанная от
%основного (см. рис. \ref{aa}  стр.   ),
%$М$ и $\mu$ - полная и приведенная массы системы;
%

%и записывая оператор ядерного
%$\beta$-перехода $H^{(\beta^-)}$ в системе центра масс в виде (\ref{H^beta}):
%\begin{eqnarray*}
%H^{(\beta^-)}=\exp{(-i\vec q_\beta\vec R_c)}\exp{( -i \vec \varkappa_\beta \vec r)}
%\sum_{j=1}^A H^{(\beta^-)}_{j}\exp{(-i\vec q_\beta\vec\xi_{j})},
%\end{eqnarray*}
%где $\hbar \vec q_\beta=\vec p_e-\vec p_\nu$, $\vec \varkappa_\beta=
%{\mu\over Am}\vec q_\beta$, $m$ - масса нуклона,
%$\vec r$ - относительная координата, $\vec R_c$~-~координата
%центра тяжести системы,
%$\xi_j$  - координата j-го нуклона, отсчитанная от центра тяжести материнского
%ядра;
%

Представляя волновые функции начального $(s=i)$ и конечного $(s=f)$
состояний системы ядро-нейтрон в виде:
\begin{eqnarray}\label{if_f}
\ket{s}= \eta^{(\pm)}(\vec k_s, \vec r)\exp(i \vec K_s \vec R_c) \ket{ \beta_s},
\end{eqnarray}
$\ket{ \beta_s}$ -- волновые функции, характеризующие внутренние состояния
материнского и дочернего ядер,
$\eta^{(\pm)}(\vec k_s, \vec r)$ -- волновые функции относительного движения
в столкновительной системе. Тогда, подставляя (\ref{if_f}) в (\ref{ds}),
после интегрирования по величинам $\vec k_f$ и $E_\nu$
для дифференциального сечения процесса СБР ядра $(A,Z)$ при столкновении
его с нейтроном получим:
\begin{eqnarray}\label{nds}
d\sigma^{(col)}_\beta&=&
{{g_v^2 \mu ^2 k_f}\over{{(2\pi)}^8 \hbar^9 c^5 k_i}} \xi_\beta k_e E_e
E^2_\nu \; dE_e \; d\varepsilon_f \; d\Omega_e \; d\Omega_{\nu}\; d\Omega_f \nonumber \\
&&\bigg| \bra {\eta^{(-)} (\vec k_f, \vec r)} \exp \bigg(i{{\vec q_\beta \cdot \vec r}\over{A+1}}\bigg)
\ket {\eta^{(+)} (\vec k_i, \vec r)}\bigg|^2.
\end{eqnarray}
Здесь
$$
E_{\nu}={\hbar^2 \over 2 \mu}\left({\vec k_i}^2 - {\vec k_f}^2 -{\mu \over M} \vec q_\beta{}^2\right)
-E_e-\Delta-\Delta_f,
$$
а величина $\xi_\beta$ определена формулой (\ref{mat_nuc}). Наша задача будет
заключаться в получении величины полного сечения $\sigma_\beta^{(col)}$.
Для решения этой задачи в формуле (\ref{nds}) можно провести вначале интегрирование
по всем угловым переменным. Тогда
\begin{equation}\label{s_b}
{d\sigma^{(col)}_\beta \over
d E_e d \varepsilon_f}
=
{{g_v^2 \mu ^2 k_f}\over{{(2\pi)}^8 \hbar^9 c^5 k_i}} \xi_\beta k_e E_e
E^2_\nu J(k_i,k_f,k_e,k_\nu)
\end{equation}
где введено обозначение:
\begin{multline}\label{J2}
J(k_i,k_f,k_e,k_\nu)=\int  d\Omega_e \; d\Omega_{\nu}\; d\Omega_f
\bigg| \bra {\eta^{(-)} (\vec k_f, \vec r)} \times\\
\times\exp \bigg(i{{\vec q_\beta \cdot \vec r}\over{A+1}}\bigg)
\ket {\eta^{(+)} (\vec k_i, \vec r)}\bigg|^2.
\end{multline}

Прежде, чем проводить расчет величины $J(k_i,k_f,k_e,k_\nu)$, сформулируем
модель для расчета волновых функций относительного движения
$\eta^{(\pm)} (k_s,\vec r)$ в системе ядро-нейтрон.

При решении
задачи о влиянии сильного взаимодействия на процесс $\beta$-распада одна из
главных проблем заключается в способе рассмотрения квантового столкновения.
Дело  в  том,  что многочастичность задачи не оставляет
надежды  на  ее достаточно точное квантовомеханическое решение,
хотя
попытки   такого   рода   предпринимались  на  базе  времязависящего
приближения   Хартри--Фока   (см.,  напр., \cite{NPh1986,NPh1987}).
Опыт  его
использования  в  задачах  эмиссии пионов и жестких фотонов показал,
что  реализация  расчетной  схемы  возможна лишь в сильно упрощенной
геометрии,  да  и  то  в  модельных задачах столкновения легких ядер
с  замкнутыми  оболочками.  Вышесказанное означает, что
теоретический   подход   к   описанию   в нашем случае
процесса ядерного $\beta$-распада  в  поле нейтрон-ядерной
столкновительной  системы  с неизбежностью должен быть модельным, по
крайней мере, в части решения проблемы нейтрон-ядерного столкновения.


В зависимости от способа учета взаимодействия нуклонов известны разные
подходы, но наибольший успех, как представляется, имеет квантово-оптическая
модель эмиссии быстрых частиц, предложенная в работах
\cite{IAF1988,kamanin,IAF1990,ECHA1991}.
Согласно этой модели, налетающая частица рассматривается в усредненном поле,
обусловленном движением нуклонов ядра-мишени. Это поле моделируется оптическим
потенциалом, а излучение частицы рассматривается как результат квантового
перехода между состояниями непрерывного спектра динамической диядерной системы.
Данная модель является полностью квантовой, поскольку
позволяет рассматривать в микроскопическом подходе как движение тяжелых ядер,
так и излучение быстрых частиц. Она  не содержит подгоночных
параметров, а использует готовые межъядерные потенциалы, полученные либо из
анализа экспериментов по упругому рассеянию, либо в микроскопических
расчетах независимым образом, и
позволяет получать замкнутые выражения для
сечений процессов, что делает расчетную схему простой и наглядной.

Итак, идея данной модели состоит в том, что взаимодействие налетающей частицы с мишенью нужно рассматривать лишь в
упругом канале, исключив из рассмотрения все неупругие. При этом гамильтониан системы сводится к двухчастичному с
эффективным потенциалом взаимодействия между частицами (\cite{Hot, Bal}). Считается, что  эмиссия   вторичной частицы
происходит  на  начальной стадии соударения, и это позволяет рассматривать процесс как периферийный, и, как следствие,
возможность выделить в межъядерном потенциале в качестве главной зависимость только от расстояния между ядрами.


Межъядерное поле при этом можно моделировать эффективным потенциалом.
Эффективный потенциал, в отличие
от операторов наблюдаемых величин,
нелокален,
неэрмитов и  зависит от энергии столкновения. Уравнение Шредингера с таким
потенциалом становится интегро-дифференциальным. Такие особенности являются
как бы ``платой'' за исключение из рассмотрения неупругих каналов.
В квантово-оптической модели, широко используемой и в атомной, и в ядерной физике,
этот нелокальный оператор, зависящий от относительной координаты, заменяется
обычно комплексной функцией этой координаты, т.е. оптическим
потенциалом взаимодействия:
\begin{equation}\label{OptPot}
U^{(\mathrm{opt})}(\varepsilon,\vec r)=V(\varepsilon,\vec r)+iW(\varepsilon,\vec r).
\end{equation}
Его параметры  являются функциями энергии относительного движения $\varepsilon$ вследствие нелокальности эффективного
потенциала. Хотя значения параметров оптического потенциала определяются при исследовании (как теоретическом, так и
экспериментальном) упругих соударений, а столкновительная эмиссия быстрых частиц является неупругим процессом, тем не
менее, процедура введения в гамильтониан эффективного потенциала может быть видоизменена включением в рассмотрение двух
сильно связанных каналов~-- входного и выходного, и исключением всех остальных. Это не затрагивает принципиальной стороны
способа моделирования нуклон-ядерного или ядро-ядерного потенциала оптическим \cite{Bal}. Таким образом,
квантово-оптическая модель,  достаточно просто и эффективно учитывает межъядерное взаимодействие.

Ее использование в работах \cite{kor7,kor8,kor9,kor10} для объяснения
наблюдаемых характеристик тормозного излучения в нейтрон-ядерных и протон-ядерных
столкновениях показало, что такой подход достаточно эффективен и при исследовании
неупругих процессов.
Итак, принимая за основу квантово-оптическую модель, будем находить
волновые функции относительного движения $\eta^{(\pm)}(k_s, \vec r)$
системы ядро-нейтрон из уравнения Шредингера с однонуклонным
оптическим потенциалом.

Используем для волновых функций $\eta^{(\pm)} (\vec k_s, \vec r)$ стандартное разложение по парциальным волнам
\cite{sobelman}:
\begin{eqnarray}\label{eta}
\eta^{(\pm)} (\vec k_s, \vec r)={(2\pi)^{3/2} \over k_s} \sum_{l,m} i^l
R_l^{(\pm)}(k_s,r){\rm Y}_{lm}(\vec r^{(0)}) {\rm Y}_{lm}^*(\vec k_s^{(0)}),
\end{eqnarray}
где $\delta_l$ -- фазы рассеяния и введено обозначение:
$\vec n^{(0)}=\vec n / \mod{\vec n}$
и использовано стандартное
обозначение для сферических функций Лапласа;
$R_l^{(\pm)}(k_s,r)$~-- радиальная волновая
функция (в общем случае комплексная). Ее явный вид, в соответствии с вышесказанным,будет  находиться численным
решением радиального уравнения Шредингера с комплексным потенциалом $U^{(opt)}(\varepsilon_s,r)$:
\begin{multline}\label{eqsh}
{1\over{r^2}}{d\over{dr}}\left(r^2{{dR_l(k_s,r)}\over{dr}}\right)-{{l(l+1)}
\over{r^2}}R_l(k_s,r)+\\
+\left(k^2_s-{{2\mu}\over{\hbar^2}}U(\varepsilon_s,r)\right)R_l(k_s,r)=0,
\end{multline}

В соответствии с общим правилом \cite{landau}, вне области действия потенциала
$U(\varepsilon_s,r)$ (оптические потенциалы являются
короткодействующими) функция $\eta^{(+)} (\vec k_s, \vec r)$ должна представлять собой
суперпозицию плоской и сферической расходящейся $R_l^{(+)}(k_i,r)$ волн,
а $\eta^{(-)} (\vec k_s, \vec r)$~--
суперпозицию плоской и сферической сходящейся $R_l^{(-)}(k_f,r)$ волн.
Это условие накладывает
определенные ограничения на вид асимптотического поведения функций
$R_l^{(\pm)}(k_s,r)$:
\begin{equation}
\label{RlAsympt}
R_l^{(\pm)}(k_s,r)\xrightarrow[r\to\infty]{}\frac{k_s}{\sqrt{2\pi}}%
\{[h_l^{(1)}(k_sr)+h_l^{(2)}(k_sr)]-(1-S_l)h_l^{(\pm)}(k_sr)\}
\end{equation}
Первые два слагаемых суммы \eqref{RlAsympt} представляют собой общий член
разложения плоской волны, а последнее~---
расходящуюся ($h_l^{(+)}\equiv h_l^{(1)}$)
или сходящуюся ($h_l^{(-)}\equiv h_l^{(2)}$) сферические волны,
$S_l$~--- коэффициент рассеяния \cite{landau}.


Как видно из (\ref{s_b}), главная проблема расчета сечения СБР
заключается в нахождении величин $J(k_i,k_f,k_e,k_\nu)$
Это можно сделать точно, если воспользоваться известным разложением
по мультиполям для экспоненты:
\begin{eqnarray}\label{exp}
exp({i\vec q_\beta \vec r \over A+1})={(4\pi)}^2 \sum_{l_e m_e}
\sum_{l_\nu m_\nu}(-1)^{l_\nu} i^{l_e+l_\nu} {\rm j_{l_e}}(\varkappa_e r)
{\rm j_{l_\nu}}(\varkappa_\nu r)\times \nonumber \\
\times {\rm Y^*_{l_e m_e}}(\vec k^{(0)}_e)
{\rm Y_{l_\nu m_\nu}}(\vec k^{(0)}_\nu){\rm Y_{l_e m_e}}(\vec r^{(0)})
{\rm Y^*_{l_\nu m_\nu}}(\vec r^{(0)}),
\end{eqnarray}
где ${\rm j_L}(x)$ -- сферическая функция Бесселя и введены обозначения:
$\vec \varkappa_a=\vec k_a/(A+1)$ $(a=e,\nu)$; $k_\nu=E_\nu/(\hbar c)$.

Подставим мультипольные разложения (\ref{eta}) и (\ref{exp}) в (\ref{J2}):
\begin{multline}\label{bigeq}
%\begin{eqnarray}
J(k_i,k_f,k_e,k_\nu)={2^4 (2\pi)^{10} \over k_i^2 k_f^2}
\sum_{l_f',m_f'}\sum_{l_f,m_f}\sum_{l_e',m_e'}\sum_{l_e,m_e}
\sum_{l_\nu',m_\nu'}\sum_{l_\nu,m_\nu}
\sum_{l_i',m_i'}\sum_{l_i,m_i}
(-1)^{l_e+l_\nu}\times\\
%
\times(-1)^{l_i+l_f}
R^{l_fl_e}_{l_il_\nu}(k_i,k_f,k_e,k_\nu)
R^{l_f'l_e'*}_{l_i'l_\nu'}(k_i,k_f,k_e,k_\nu)
{\rm Y}^*_{l_im_i}(\vec k_i^{(0)}){\rm Y}_{l_im_i}(\vec k_i^{(0)})\times\\
%
\times\int d \Omega_e d\Omega_\nu d\Omega_f
{\rm Y}^*_{l_em_e}(\vec k_e^{(0)}) {\rm Y}_{l_e'm_e'}(\vec k_e^{(0)})
{\rm Y}_{l_\nu l_\mu}(\vec k_\nu^{(0)}){\rm Y}_{l_\nu' l_\mu'}^*(\vec k_\nu^{(0)})
{\rm Y}_{l_fm_f}(\vec k_f^{(0)})\times\\
%
\times{\rm Y}_{l_f'm_f'}^*(\vec k_f^{(0)})
\int d\Omega_r {\rm Y}_{l_fm_f}^*(\vec r^{(0)}){\rm Y}_{l_em_e}(\vec r^{(0)})
{\rm Y}_{l_im_i}(\vec r^{(0)}){\rm Y}^*_{l_\nu l_\mu}(\vec r^{(0)})\times\\
\times\int d\Omega_r {\rm Y}_{l_f'm_f'}^*(\vec r^{(0)}){\rm Y}^*_{l_e'm_e'}(\vec r^{(0)})
{\rm Y}_{l_i'm_i'}^*(\vec r^{(0)}){\rm Y}_{l_\nu' m_\mu'}^*(\vec r^{(0)})
\end{multline}
где
\begin{eqnarray}\label{R}
{\rm R}^{l_f l_e}_{l_i l_\nu}(k_i, k_f, k_e, k_\nu)=
\int{ {\rm R}^*_{l_f}(k_f,r) {\rm j}_{l_e}(\varkappa_e r){\rm j}_{l_\nu}
(\varkappa_\nu r)
{\rm R}_{l_i}(k_i,r) r^2 dr}.
\end{eqnarray}

Проинтегрируем вначале по угловым переменным вектора $\vec r$, используя   выражение произведения двух сферических
функций одного и того же аргумента в виде линейной комбинации сферических функций того же аргумента (Е,1) \cite{davydov}
и интегральную формулу для сферических функций Лапласа (Е,4)\cite{davydov}. В результате получим:
\begin{eqnarray}\label{yyyy}
&&\int{{\rm Y^*_{l_f m_f}}(\vec r^{(0)})
{\rm Y_{l_e m_e}}(\vec r^{(0)}){\rm Y_{l_i m_i}}(\vec r^{(0)})
{\rm Y^*_{l_\nu m_\nu}}(\vec r^{(0)}) d\Omega_r}=\nonumber\\
&&=\sum^{l_e+l_\nu}_{l=\mod{l_e-l_\nu}}{{((2l_e+1)(2l_\nu+1)(2l_i+1)
(2l_f+1))^{1/2}}
\over{4\pi(2l+1)}}\nonumber\\
&&(-1)^{m_e+m_f+l_i+l_f-l}
C_{l_e l_\nu 0 0 }^{l 0} C_{l_i l_f 0 0 }^{ l 0 }
C_{l_e l_\nu m_e -m_\nu}^{ l (m_e-m_\nu)}
C_{l_i l_f -m_i m_f}^{l (m_e - m_\nu)}
\end{eqnarray}
Здесь $C^{LM}_{l_1m_1l_2m_2}$-коэффициенты векторного сложения
моментов количества движения

Выберем направление оси квантования вдоль вектора $\vec k_i$. Тогда:
\begin{eqnarray}\label{Y0}
\int{{\rm Y_{l_i m_i}}(\theta_i, \phi_i)}= \sqrt{{{2l_i+1}\over{4\pi}}} \delta_{m_i 0}.
\end{eqnarray}


Проводя теперь интегрирование по угловым переменным в (\ref{bigeq})
и учитывая (\ref{yyyy}) и (\ref{Y0}), получим:
\begin{multline}\label{mat2}
J(k_i,k_f,k_e,k_\nu)={{2{(2\pi)}^7}\over{k_i^2 k_f^2}}
\sum_{ll'} \sum_{l_i l'_i} \sum_{l_f m_f} \sum_{l_e m_e} \sum_{l_\nu m_\nu}
(-1)^{l_i-l-l'} i^{l_i+l'_i} e^{i\delta_{l_i}-i\delta_{l'_i}}\times\\
\times C^{l 0}_{l_e 0 l_\nu  0}
C^{l 0}_{l_i 0 l_f  0} C^{l' 0}_{l_e 0 l_\nu  0} C^{l' 0}_{l'_i 0 l_f  0}
C^{l (m_e - m_\nu)}_{l_e m_e l_\nu -m_\nu}  C^{l (m_e - m_\nu)}_{l_i 0 l_f  m_f}\times\\
\times
C^{l' (m_e - m_\nu)}_{l_e m_e l_\nu -m_\nu}  C^{l' (m_e - m_\nu)}_{l'_i 0 l_f m_f}
{{(2l_f+1)(2l_i+1)(2l'_i+1)}\over{(2l+1)(2l'+1)}}\times\\
\times (2l_e+1)(2l_\nu+1) {\rm R}^{l_f l_e}_{l_i l_\nu}(k_i, k_f, k_e, k_\nu){\rm R}^{l_f l_e *}_{l'_i l_\nu}(k_i, k_f, k_e, k_\nu).
\end{multline}

Используя стандартную технику суммирования коэффициентов Клеб\-ша-Гор\-до\-на,
можно провести суммирование по магнитным квантовым числам
$m_e$, $m_\nu$, $m_f$. Тогда,
\begin{multline}\label{S2}
\sum_{m_f m_e m_\nu}
C^{l (m_e - m_\nu)}_{l_e m_e l_\nu -m_\nu}  C^{l (m_e - m_\nu)}_{l_i 0 l_f  m_f}
C^{l' (m_e - m_\nu)}_{l_e m_e l_\nu -m_\nu}  C^{l' (m_e - m_\nu)}_{l'_i 0 l_f m_f}
=\sum_j(-1)^{l+l'}\delta_{l_i l'_i}\times\\
\times{{(2l+1)(2l'+1)(2j+1)}\over{(2l_i+1)}}
{\rm W}(l_e l_\nu l_i l_f;l j) {\rm W}(l_e l_\nu l_i l_f;l' j),
\end{multline}
где ${\rm W}(l_1 l_2 l_3 l_4 ; l_5 l_6)$ -- коэффициент Рака,
$\delta_{k k'}$ -- символ Кронекера.

Учитывая (\ref{S2}) и проведя суммирование коэффициентов Рака, окончательно
получим:
\begin{multline}\label{2mat2}
J(k_i,k_f,k_e,k_\nu)
={{2{(2\pi)}^7}\over{k_i^2 k_f^2}}
\sum_{l_i} \sum_{l_f} \sum_{l_e} \sum_{l_\nu} \sum_{l}
(2l_i+1)(2l_f+1)(2l+1)
%[l_i l_f l_e l_\nu l]
\times\\
\times(2l_e+1)(2l_\nu+1)
{\3jm{l_e}{l_\nu}{l}{0}{0}{0}}^2 {\3jm{l_i}{l_f}{l}{0}{0}{0}}^2
\left|{ \rm R}^{l_f l_e}_{l_i l_\nu}(k_i, k_f, k_e, k_\nu)\right|^2,
\end{multline}
Здесь $\quad \3jm{j_1}{j_2}{j_3}{m_1}{m_2}{m_3}$ -- $3jm$-символы.

С учетом (\ref{2mat2}) полное сечение процесса столкновительного
$\beta$-распада с переходом из основного состояния материнского ядра $(A,Z)$
в состояние $\ket{\beta_f}$ дочернего ядра $(A,Z+1)$ с энергией
$\Delta_f$
(предполагается, что квантовые правила отбора для $\beta^-$-перехода
разрешенного типа при этом
выполняются) принимает вид:
\begin{multline}\label{ndse}
\sigma^{(col)}_\beta(\varepsilon_i, E^f_\Delta)=
{{g_v^2 \mu ^2}\over{\pi \hbar^9 c^5 k_i^3}} \xi_\beta
\int_{0}^{\varepsilon_i-E^f_\Delta-m_ec^2}{{d\varepsilon_f}\over{k_f}}\times\\
\times\int_{m_e c^2}^{\varepsilon_i-E^f_\Delta-\varepsilon_f}{k_e E_e E^2_\nu
{\rm J}(k_i, k_f, k_e, k_\nu) {\rm F_0}(Z,E_e)\; dE_e}.
\end{multline}
Здесь $E^f_\Delta\equiv \Delta_f+\Delta$. Влияние кулоновского поля дочернего ядра на $\beta$-электрон учтено введением в
подинтегральное выражение (\ref{ndse}) кулоновской функции Ферми ${\rm F_0}(Z,E_e)$ \cite{aiz}. Для нее в расчетах
использовалось следующее выражение
\begin{equation}\label{fun_F}
{\rm F_0}(Z,E_e)=2(1+\gamma_1)\left({{2 p_e R_0}\over{\hbar}}\right)^{2(\gamma_1-1)}\e^{\pi \eta}
\left|{{\Gamma(\gamma_1+i \eta )}\over{\Gamma(2\gamma_1+1)}}\right|^2,
\end{equation}
где $\gamma_1=(1-(\alpha_e Z)^2)^{1/2}$, $\eta=\alpha_e Z E_e/(c p_e)$,
$R_0$ -- радиус ядра.


 Расчет полного сечения процесса столкновительного $\beta$-распада
упрощается, если воспользоваться ``дипольным'' приближением (пренебречь в формуле
(\ref{nds})
зависимостью от $\vec q_\beta$), т.е. положить
\begin{eqnarray}\label{dip}
\exp \bigg(i{{\vec q_\beta \cdot \vec r}\over{A+1}}\bigg)\approx 1,
\qquad \qquad {{\hbar^2 q_\beta^2}\over{2M}}\approx 0.
\end{eqnarray}

В этом случае:
\begin{eqnarray}\label{Jdip}
{\rm J}(k_i, k_f,k_e,k_\nu)={2(2\pi)^7 \over k_i^2 k_f^2}\sum_{l} (2l+1)\left|R_l(k_i,k_f)\right|^2
\end{eqnarray}
где
\begin{equation}
R_l(k_i,k_f)=\int_{0}^{\infty}R^*_l(k_f,r)R_l(k_i,r) r^2 dr
\end{equation}


\section{Вычисление радиальных волновых \\ функций и интегралов.}

Как видно из формул (\ref{2mat2})-(\ref{ndse}),  величина сечения процесса СБР
определяется радиальным интегралом
${\rm J}(k_i, k_f, k_e, k_\nu)$. Для их расчета на первом этапе
необходимо найти радиальные волновые функции
${\rm R}_{l_s}(k_s, r)$ в
разложении (\ref{eta}),
решая уравнение
Шредингера (\ref{eqsh}) с нейтронным оптическим потенциалом.

Сделаем в уравнении (\ref{eqsh}) подстановку:
\begin{eqnarray}\label{chi}
{\rm R}_{l_s}(k_s,r) \equiv {{\chi_{l_s}(k_s,r) + i \psi_{l_s}(k_s,r)}\over r},
\end{eqnarray}
разделяя вещественную $\chi_{l_s}(k_s,r)$ и мнимую $\psi_{l_s}(k_s,r)$
компоненты функций ${\rm R}^{\pm}_{l_s}$.

Тогда вместо (\ref{eqsh}) получается система уравнений:
\begin{eqnarray}\label{eq2}
\chi''_{l_s}(k_s,r)&=&{{l(l+1)}\over{r^2}}\chi_{l_s}(k_s,r)-k^2_s \chi_{l_s}(k_s,r)+\nonumber\\
&&+{{2\mu}
\over{\hbar^2}}\{\Re U {\cdot} \chi_{l_s}(k_s,r)-\Im U {\cdot} \psi_{l_s}(k_s,r)\},\nonumber\\
\psi''_{l_s}(k_s,r)&=&{{l(l+1)}\over{r^2}}\psi_{l_s}(k_s,r)-k^2_s \psi_{l_s}(k_s,r)+\nonumber\\
&&+{{2\mu}
\over{\hbar^2}}\{\Re U {\cdot} \psi_{l_s}(k_s,r)+\Im U {\cdot} \chi_{l_s}(k_s,r)\}.
\end{eqnarray}
Очевидно, что в качестве начального условия для функций $\chi_{l_s}(k_s,r)$
и $\psi_{l_s}(k_s,r)$ следует взять $r^{l+1}$, поскольку в этих
уравнениях (\ref{eq2})
при $r \to 0$ основную роль играет центробежный член.

Как известно, оптические потенциалы являются короткодействующими. Действие
 ядерного потенциала $U(r)$ практически исчезает на расстоянии $\tilde R_0$,
в несколько раз превышающем радиус ядра, и в области $r>\tilde R_0$ решения уравнений
принимают вид:
\begin{equation}
\label{Free}
\begin{split}
\chi_{l_s}(k_s, r)&=k_s r\,[C_1\,j_l(k_s, r)+C_2\, y_l(k_s, r)];\\
\psi_{l_s}(k_s, r)&=k_s r\,[B_1\,j_l(k_s, r)+B_2\, y_l(k_s, r)].
\end{split}
\end{equation}
Здесь  $j_l(k_s r)$, $y_l(k_s r)$~-- сферические функции Бесселя и
Неймана.

Функции $\chi_{l_s}(k_s, r)$ и $\psi_{l_s}(k_s, r)$
необходимо нормировать. Для этого следует потребовать, чтобы при
$r\to\infty$ функции $R_{k,l}^{(\pm)}(r)$ имели асимптотику \eqref{RlAsympt}.
Тогда
\begin{equation}
\label{Norm}
R_{k,l}^{(\pm)}(r)=\sqrt{\frac{2}{\pi}}\cdot
\frac{\chi_{k,l}(r)+i\psi_{k,l}(r)}{[C_1\mp B_2+i(B_1\pm C_2)]r}.
\end{equation}

Для проведения численных расчетов необходимо знать явный вид оптического
потенциала $U^{(\mathrm{opt})}(\varepsilon,\vec r)$ (\ref{OptPot}).


Получению
параметров оптических потенциалов  на основе анализа экспериментов по
упругому нейтрон-ядерному рассеянию, было посвящено много работ \cite{PhR1969, PhR1971, PhR1981, NPh1964, PhR1987}.
Как правило, в этих работах энергия столкновения фиксирована, что делает такие
потенциалы неудобными при исследовании неупругих процессов,
когда энергия конечного состояния отличается от начального.
Поскольку в задаче СБР из-за эмиссии  лептонной пары проводится
интегрирование по энергии конечного состояния нейтрон-ядерной системы (смотри
формулу (\ref{ndse}), желательно иметь такое выражение для
оптического потенциала, в котором зависимость его параметров от энергии
столкновения приведена в аналитическом виде. В интересующей нас области
энергий столкновения выражение такого рода приведено в работе \cite{perey}
(а также и в обзоре \cite{bechetti}). Как результат компилляции различных
экспериментальных данных по упругому нейтрон-ядерному рассеянию
с фиксированной энергией столкновения.
Согласно \cite{perey}, представим $U^{(opt)}(r,\varepsilon_s)$ в следующем виде:
\begin{eqnarray}\label{opt}
U(r,\varepsilon_s)=-V(\varepsilon_s)f(x_0)-i\left[W(\varepsilon_s)f(x_W)-
4W_D(\varepsilon_s){{d}\over{dx_D}}f(x_D)\right].
\end{eqnarray}
Здесь
\begin{equation}\label{param}
f(x_j)=(1+\e^{x_j})^{-1}, \quad x_j=(r-R_j)/a_j, \quad R_j=r_j A^{1/3} \qquad (j=0,W,D).
\end{equation}
В (\ref{opt})
параметры $V$, $W$, $W_D$, $r_i$, $a_i$
являются функциями относительной энергии $\varepsilon_s$ и массового числа А:
\begin{itemize}
\item[1)] для случая $\varepsilon_s<15$ МэВ:
\begin{equation}\label{1param}
\begin{split}
& V=47,01-0,267\varepsilon_s -0,0018\varepsilon_s^2;\\
&W_D=9,52-0,053\varepsilon_s;\\
&r_0=1,322-7,6\times 10^{-4}A+4\times 10^{-6}A^2-8\times 10^{-9}A^3;\\
&r_D=1,266-3,7\times 10^{-4}A+2\times 10^{-6}A^2-4\times 10^{-9}A^3;\\
&a_0=0,66; \qquad a_D=0,48;
\end{split}
\end{equation}

\item[2)] для случая $\varepsilon_s>15$ МэВ:
\begin{equation}\label{2param}
\begin{split}
&V=56,3-0,32\varepsilon_s -24(N_s-Z_s)/A;\\
&W=0,22\varepsilon_s-1,56;\\
&W_D=13-0,25\varepsilon_s -12(N_s-Z_s)/A;\\
&r_0=1,17, \quad r_W=r_D=1,26;\\
&a_0=0,75; \qquad a_W=a_D=0,58.
\end{split}
\end{equation}
\end{itemize}
Здесь $s=i,f$; $Z_s, N_s$ -- числа протонов и нейтронов соответственно
в материнском $(s=i)$ и дочернем $(s=f)$ ядрах; единица измерения энергии -- МэВ,
длины -- ферми; параметры $W$ и $W_D$ следует положить равными нулю,
если они становятся отрицательными.

В наших расчетах мы будем использовать эмпирические выражения
(\ref{opt}-\ref{2param}), и в этом смысле для нас параметры
физического потенциала не являются подгоночными.
В области действия оптического потенциала ($r<\tilde R_0$)
расчет радиальных интегралов $R_{l_il_\nu}^{l_fl_e}(k_i,k_f,k_e,k_\nu)$ выполнялся непосредственно
с функциями, найденным численным решением системы уравнений (\ref{eq2})
и нормированными согласно условию  (\ref{Norm}).
Вне этой области, т.е. при $r>\tilde R_0$  интегралы
представляют собой линейные комбинации интегралов вида
$\intl_{\tilde R_0}^{\infty} f_{l_f}(k_fr)
j_{l_e}(\varkappa_e r)j_{l_\nu}(\varkappa_\nu r) j_{l_i}(k_i r)r^2dr $
и
$\intl_{\tilde R_0}^{\infty} f_{l_f}(k_fr)
j_{l_e}(\varkappa_e r)j_{l_\nu}(\varkappa_\nu r) y_{l_i}(k_i r)r^2dr $
, которые
сходятся, но не абсолютно $\\(f_{l_f}(k_fr)=j_{l_f}(k_fr)$  или $y_{l_f}(k_fr)$).
Проблема  улучшения их сходимости решалась методом,
предложенным в \cite{PhR1970}.

Рассмотрим вначале интеграл
\begin{equation}\label{intj}
I=\intl_{\tilde R_0}^{\infty} f_{l_f}(k_fr)
j_{l_e}(\varkappa_e r)j_{l_\nu}(\varkappa_\nu r) j_{l_i}(k_i r)r^2dr\equiv
\int\limits_{\tilde R_0}^\infty g(r)dr.
\end{equation}
Множитель $j_{l_i}(k_i r)$, как следует из закона сохранения энергии, осциллирует наиболее быстро. Представим сферическую
функцию Бесселя первого рода $j_{l_i}(k_i r)$ через сферические функции третьего рода \cite{abramc}: $j_{l_i}(k_i
r)=(h^{(1)}_{l_i}(k_i r)+ h^{(2)}_{l_i}(k_i r))/2.$ Продолжим подинтегральную функцию с множителем $h^{(1)}_{l_i}(k_i r)$
в верхнюю комплексную полуплоскость, а с множителем $h^{(2)}_{l_i}(k_i r)$ -- в нижнюю, и поскольку в комплексной
полуплоскости
$z\geqslant \tilde R_0$ подинтегральная функция  не имеет полюсов,
деформируем контур
интегрирования от $\tilde R_0$ до $\infty$ соответственно
в отрезки $\tilde R_0 A$ и дугу
${}^{\smallsmile}\!\!AC$ и $\tilde R_0 B$ и дугу ${}^{\smallsmile}\!\!BC$ (см. \ref{COUNT}),
а затем устремим радиусы
дуг ${}^{\smallsmile}\!\!AC$ и ${}^{\smallsmile}\!\!BC$ к бесконечности.
Кроме того, согласно
лемме Жордана, $\intl_{{}^{\smallsmile}\!\!AC}g(z)\,dz\to0$ и
$\intl_{{}^{\smallsmile}\!\!BC}g(z)\,dz\to0$, а потому
\begin{equation}
\label{dy}
\intl_{\tilde R_0}^{\infty} g(r)\,dr=
i\intl_0^{\infty}g(\tilde R_0 \pm iy)\,dy.
\end{equation}
При $y>0$ подинтегральная функция имеет асимптотику затухающей экспоненты.
Таким образом, переход в комплексную плоскость делает интеграл
(\ref{intj}) сходящимся абсолютно и позволяет рассчитывать его численно
по формуле \eqref{dy}.


Вычисление интеграла
$\intl_{\tilde R_0}^{\infty} f_{l_f}(k_fr)
j_{l_e}(\varkappa_e r)j_{l_\nu}(\varkappa_\nu r) y_{l_i}(k_i r)r^2dr $
проводится аналогичным образом с использованием представления функций
$y_{l_i}(k_i r)$ через сферические функции Бесселя третьего рода:
$y_{l_i}(k_i r)=(h^{(1)}_{l_i}(k_i r)-h^{(2)}_{l_i}(k_i r))/(2i).$

\unitlength=1cm
\begin{figure}
\begin{picture}(13,12)
\put(0,12){\special{em:graph contour.pcx}}
\end{picture}
\caption{Контур интегрирования при вычислении радиальных интегралов}
\label{COUNT}
\end{figure}

%\begin{figure}
%\vspace{18 true cm}
%\caption{Контур интегрирования при вычислении радиальных интегралов}
%\label{COUNT}
%\end{figure}



\section{Результаты расчетов сечений процесса столкновительного $\beta$-распада,
инициированного столкновениями с нейтронами.}

С учетом особенностей вычисления радиальных волновых функций и радиальных
интегралов, подробно рассмотренных в предыдущем разделе, были проведены расчеты
полного сечения процесса столкновительного $\beta$-распада, инициируемого
нейтрон-ядерными столкновениями. Исследование зависимости  полного сечения
процесса $\sigma^{(col)}_\beta$ от начальной энергии столкновения $\varepsilon_i$
и пороговой энергии $E^f_\Delta$ было проведено для ядра $^{162}Dy$.
Результаты расчетов для значений $E^f_\Delta$, равных 1.0, 2.0 и 2.5 МэВ
представлены на рис. \ref{EF}  . Как видно, величина полного сечения  в диапазоне
начальной энергии столкновения от 4 до 15 МэВ растет с увеличением
$\varepsilon_i$, причем сечение больше для меньших значений пороговой
энергии. Как и следует ожидать, при $\varepsilon_i \gg E^f_\Delta$
различие между величинами сечений становится незначительным.

\begin{figure}
\vspace{18 true cm}
\caption{{ Полное сечение  $\sigma^{(col)}_\beta$ СБР как функция
 энергии относительного движения ядер $\varepsilon_i$ стабильного изотопа
$^{162}Dy$. Пороговые энергии $E^f_\Delta= 1.0 \; МэВ$ (1), $E^f_\Delta= 2.0 \; МэВ$ (2),
$E^f_\Delta= 2.5 \; МэВ$ (3). Значения сечения приведены
в единицах $\xi_\beta$.}}
\label{EF}
\end{figure}


Зависимость полного сечения процесса столкновительного $\beta$-распада от
начальной энергии столкновения $\varepsilon_i$ при фиксированном значении
пороговой энергии $E^f_\Delta$  представлена на рис. \ref{EF25} для следующих изотопов:
$^{74}Ge, \; ^{106}Pd, \; ^{152}Gd, \; ^{162}Dy$.

\begin{figure}
\vspace{18 true cm}
\caption{{ Полное сечение  $\sigma^{(col)}_\beta$ процесса СБР как функция
 энергии относительного движения ядер $\varepsilon_i$ стабильных изотопов:
$^{152}Gd$ (1), $^{162}Dy$ (2), $^{106}Pd$ (3) и $^{74}Ge$ (4). Пороговая энергия $E^f_\Delta= 2.5 \; МэВ$.
Значения сечения приведены
в единицах $\xi_\beta$.}}
\label{EF25}
\end{figure}



Полученные величины полных сечений процесса СБР оказываются на несколько
порядков больше, чем в случае кулоновских столкновений стабильных ядер с
протонами тех же энергий. На рис. \ref{EF4}  представлены сечения столкновительного
$\beta$-распада ядра $^{84}Sr$,  стимулированного нейтрон-ядерными (1) и
протон-ядерными кулоновскими (2) столкновениями. Видно, что особенно большое различие
наблюдается при энергиях столкновения, близких к пороговой энергии
$E^f_\Delta$.


\begin{figure}
\vspace{18 true cm}
\caption{{ Сравнение зависимости сечений процесса СБР    $\sigma^{(col)}_\beta$ от энергии относительного
движения ядер $\varepsilon_i$ ядра $^{84}Sr$, стимулированного нейтрон-ядерными
 (1) и ядро-ядерными (2)   столкновениями (пороговая энергия $\Delta=2.68\; МэВ$).}}
\label{EF4}
\end{figure}


Расчеты сечений по формулам (\ref{ndse}), (\ref{2mat2}) и (\ref{Jdip})
показали, что использование ``дипольного'' приближения может изменить
величины сечений в пределах  $10 \%$, особенно когда  $\varepsilon_i \geq 15$ МэВ.


Сформулируем основные результаты, полученные в данной главе.


С использованием мультипольных разложений по относительной координате оператора
перехода и волновых функций непрерывного спектра,
получено замкнутое выражение для полного сечения процесса столкновительного
$\beta$-распада, стимулированного нейтрон-ядерными столкновениями.
Волновая функция относительного движения в нуклон-ядерной системе находится
непосредственно из
уравнения Шредингера с оптическим потенциалом.

Показано, что при нейтрон-ядерных столкновениях величины полных сечений
существенно увеличиваются по сравнению с кулоновскими столкновениями
этих ядер с протонами тех же энергий (примерно на $7 \div 12$ порядков
в зависимости от энергии столкновения).


Выявлена чувствительность полных сечений процесса столкновительного
$\beta$-распада, стимулированного сильными нейтрон-ядерными столкновениями
к величине $E_\Delta^f$
при малых энергиях относительного движения $\varepsilon_i$.
При $\varepsilon_i \gg E_\Delta^f$, как и следовало ожидать, различие между
величинами сечений (для разных $E_\Delta^f$) становится незначительным.


Основные результаты, полученные в данной главе, опубликованы в работах
\cite{K8, K10, K11}.

%EOF
