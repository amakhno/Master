
\chapter*{Введение}
\addcontentsline{toc}{chapter}{Введение} \label{intro}



%Одним из главных направлений экспериментальной и теоретической
%физики последнего времени является изучение физических
%процессов, стимулированных ядро-ядерными  столкновениями при
%промежуточных энергиях (20-150 Мэв/нуклон).

Исследование физических процессов, стимулированных ион-ионными столкновениями, составляет одно из главных направлений
экспериментальной и теоретической физики последнего времени. Это обусловлено, с одной стороны, введением в действие
соответствующих ускорителей тяжелых ионов в ряде ведущих ядерных центров, а, с другой - ценностью получаемой в процессе
исследования информации о динамике ядро-ядерного соударения, свойствах межъядерного взаимодействия,  деталях структуры
сталкивающихся частиц, механизме реакции и пр. (см., например, обзоры \cite{braun, batkinech}). Наряду с этим оставалось
пока малоизученным такое возможное направление исследований в этой области, как стимулирование естественной и
искусственной радиоактивности ядер в нуклон-ядерных и ядро-ядерных столкновительных системах. Столкновения нуклонов  или
ядер с ядрами при достаточной относительной энергии и выполнении необходимых квантовых правил отбора могут ускорить
$\gamma$-разрядку метастабильных ядерных состояний, $\alpha$- или $\beta$-распад естественно активных изотопов, а также
сделать возможными новые явления:
$\alpha$- или $\beta$-распад стабильных изотопов.
В известной мере аналогом последних является достаточно хорошо изученное
ядерное тормозное излучение, т.е. процесс, в котором также столкновения
$\gamma$-стабильных частиц рождают электромагнитное излучение.
Все выше перечисленные процессы могут рассматриваться как следствие свободно-свободных переходов в динамической
столкновительной системе. Не исключено, что их исследование позволит получить не только более ясную физическую картину
столкновительного процесса, но и выявить такие особенности распадных процессов, которые ранее были не известны в
естественных условиях. Во-первых, само столкновение ядерных частиц при одновременном осуществлении радиоактивного
процесса, переходит в разряд неупругих. Это открывает новые возможности для изучения структурных характеристик
сталкивающихся частиц и особенностей межъядерного взаимодействия \cite{minin,huskivadze}. Во-вторых, за счет столкновения
существенно расширяются как энергетический диапазон, так и интервал переданных импульсов, что также создает необычные
условия для реализации распадных процессов и может выявить новые их закономерности.


Настоящая диссертация посвящена исследованию $\beta$-раcпада
$\beta$-ста\-биль\-ных
ядер, стимулированного нуклон-ядерными и ядро-ядерными столкновениями. Впервые возможность столкновительного
$\beta$-распада \linebreak (СБР) стабильных изотопов была рассмотрена в работе \cite{batkin}, где этот процесс
исследовался для случая ядро-ядерного столкновения. Предполагалось, что $\beta$-распад стабильного ядра запрещен только
законом сохранения энергии, а по спиновым и изоспиновым характеристикам состояний материнского и дочернего ядер запрета
нет. Было показано, что процесс столкновения
$\beta$-стабильного ядра с другим ядром
приводит к его $\beta$-распаду, если энергии столкновения достаточно для
преодоления энергетического порога, препятствующего естественному
$\beta$-переходу.
Примечательно, что для осуществления СБР стабильного ядра кулоновский барьер,
в принципе, не является помехой, так что процесс СБР представляет собой еще
один возможный канал ядерных превращений, открытый при относительно малых
энергиях сталкивающихся частиц.
Это позволяет рассмотреть случай, когда кинетическая энергия относительного
движения ядер меньше высоты кулоновского барьера, и взаимодействие ядер считать
чисто кулоновским, исключив из рассмотрения возможность превращения
(в результате столкновения) материнского ядра $(A,Z)$ в изобарное дочернее
ядро $(A,Z+1)$ по каналам сильного взаимодействия. В противном случае слабый
эффект СБР в экспериментах по его обнаружению маскировался бы значительным
выходом дочерних ядер $(A,Z+1)$ за счет чисто ядерных процессов.

В \cite{batkin} расчет сечения СБР проводился в борновском приближении, которое пригодно для получения первоначальных
сведений о величинах сечений, но не дает полной картины при описании процесса столкновительного
$\beta$-распада. Поэтому в настоящей работе расчет сечения столкновительного
$\beta^-$-распада ядер в чисто кулоновском поле выполняется
методом искаженных волн с точными кулоновскими функциями.

Большой интерес представляет также процесс столкновительного
$\beta$-распада, инициированного нейтрон-ядерными столкновениями.
В этом случае появляется возможность учитывать помимо кулоновского
сильное взаимодействие в столкновительной системе, что, в свою очередь,
позволяет расширить диапазон рассматриваемых энергий и использовать рост
сечения СБР с увеличением столкновительной энергии. В настоящей работе
получено выражение для дифференциального сечения процесса СБР стабильного
ядра, стимулированного нейтрон-ядерным столкновением, и рассчитано полное
сечение такого процесса в зависимости от относительной энергии нейтрона и
высоты пороговой энергии для столкновительного $\beta$-распада.

В теоретическом рассмотрении процесса столкновительного $\beta$-рас\-па\-да стабильных изотопов, стимулированного
нейтрон-ядерными столкновениями, за основу мы принимаем квантово-оптическую модель - микроскопическую модель, в которой
на основе квантовой механики рассматриваются одновременно и процесс столкновения, и процесс рождения частиц. Эта модель с
успехом применялась для описания эмиссии быстрых частиц в ион-ионных столкновениях: $\gamma$-квантов \cite{kamanin},
подпороговых пионов \cite{batkinech} и позитронов \cite{minin}. Согласно этой модели эмиссия быстрой вторичной частицы
происходит на начальной стадии соударения, т.е. механизм появления вторичных частиц аналогичен ядерному тормозному
излучению (в квантовой постановке задачи). Как известно, в последнем случае электромагнитное излучение возникает там, где
межъядерный потенциал неоднороден и сила максимальна. По аналогии, для эмиссии вторичных частиц, не тольно фотона, в том
же механизме будут существенны поверхностные области сталкивающихся ядер, где градиент ядерного потенциала максимален.
Отсюда предполагается главная роль именно периферийных столкновений и, как следствие, возможность выделить в межъядерном
потенциале в качестве главной зависимость только от расстояния между ядрами. Эту зависимость можно апроксимировать,
моделируя межъядерное поле оптическим потенциалом, и тем самым ликвидировать подгоночные параметры (используются
известные оптические потенциалы, полученные из экспериментов по ядро-ядерному и нуклон-ядерному рассеянию). При таком
подходе процесс столкновительного $\beta$-распада одного из столкновительных партнеров  ничем не отличается от процесса,
например, эмиссии электромагнитного излучения, только теперь электромагнитная вершина заменяется на слабую. Диаграмма
процесса столкновительного $\beta$-распада представлена на рис.~\ref{DSBR}.

\begin{figure}
\vspace{5 true cm}
\caption{Диаграмма процесса столкновительного \be-распада. Индексы
i и f отмечают начальное и конечное состояние системы.}
\label{DSBR}
\end{figure}


 Итак сильное
взаимодействие нейтрона с ядром мы учитываем в рамках оптической модели и волновую функцию относительного движения в
столкновительной системе находим численным решением соответствующего уравнения Шредингера. Эта расчетная схема была ранее
опробована в \cite{karpov}, где решалась задача поиска ядер-кандидатов для прямого наблюдения процесса СБР в
нейтрон-ядерных столкновениях. Однако в данном случае мы расчетную схему усовершенствуем за счет отказа от "дипольного"
приближения при расчете матричного элемента процесса, когда выполняется интегрирование по относительной координате. Это
может оказаться существенным при рассмотрении широкого интервала столкновительных энергий. Соответственно энергия,
уносимая слабым полем, тоже может оказаться значительной и "дипольное" приближение будет недостаточным. Кроме того, в
отличие от \cite{karpov}, сами расчеты выполнялись для значительно более широкого круга стабильных нуклидов, а именно
тех, которые могут рассматриваться как праматеринские при решении проблемы нуклеосинтеза обойденных ядер в звездном
веществе (эта проблема будет обсуждена ниже).


Рассчитанные сечения процесса СБР стабильных ядер имеют величины, характерные для процессов с участием слабого
взаимодействия. Это пока слишком мало, чтобы осуществить прямое наблюдение явления СБР на действующих ускорителях тяжелых
ионов (или нуклонов), хотя в некоторых случаях влияние фоновых полей может быть уменьшено. Так в \cite{karpov} была
предложена схема эксперимента, в которой, в частности предполагается стимулировать столкновительный $\beta$-переход в
метастабильное состояние дочернего ядра с последующей идентификацией $\gamma$- или конверсионного перехода в основное
состояние. Даже в этом случае, к сожалению, современные ускорители не могут пока создать нейтронные пучки необходимой
интенсивности.

Хотя прямое наблюдение столкновительного $\beta$-распада, как выяснилось, пока маловероятно, существуют  возможности
получения косвенных свидетельств реальности процесса. Одна из них рассмотрена в \cite{huskivadze}, где  была предложена
кинетическая модель естественной
$\beta^+$-активности
ядер. В ее основу в качестве элементарного процесса был положен
столкновительный $\beta^+$-распад протона по такому же механизму,
как и для стабильных ядер. Выяснилось, что расчет по этой модели
периодов полураспада всех известных $\beta^+$-активных изотопов
с $A<100$ (их около 30) дает результаты, неплохо воспроизводящие
сильно нерегулярный  ход экспериментальной кривой.

Другая возможность получения косвенных свидетельств -- исследование
столкновительных процессов в звездном веществе в условиях высоких температур,
и, в частности,
в плане решения старой астрофизической проблемы происхождения обойденных
изотопов (иначе, $p$-ядер).
В этом случае малость сечений процесса СБР может быть
скомпенсирована или большими временными промежутками (на стадии
квазиравновесного состояния звездного вещества), или интенсивными
столкновениями частиц (при взрыве сверхновых звезд).

До настоящего времени, по-существу, в ядерной астрофизике остается
одной из нерешенных  проблема происхождения обойденных
ядер. Обойденные изотопы
(их насчитывается больше 30)~-~это наиболее богатые протонами $\beta$-стабильные
ядра, тяжелее железа. Их распространенность в среднем на два-четыре порядка ниже, чем
у соседних стабильных изобар главной последовательности.
Именно проблеме образования обойденных ядер
по механизму СБР, решение которой представляет и самостоятельный интерес,
частично посвящено данное исследование.


Как известно, источником большинства ядер  являются последовательные
ядерные реакции, протекающие в звездах, а именно: водородное, гелиевое,
углеродное, неоновое, кислородное и кремниевое горение,
отвечающие за образование ядер вплоть до элементов железного пика;
и $r$- и  $s$-процессы, основанные на механизме нейтронного захвата
с последующим (или одновременным) $\beta^-$-распадом образовавшихся изотопов,
отвечающие за нуклеосинтез средних и тяжелых ядер.
Наиболее распространенные изотопы тяжелее железа сформировались, очевидно, в
недрах массивных звезд в результате таких последовательных реакций захвата
свободных нейтронов. Ряд характерных особенностей хода кривой
распространенности этих тяжелых ядер указывает на то, что процесс их
построения должен протекать достаточно эффективно как на сравнительно
продолжительной равновесной стадии эволюции звезд в условиях малых
интенсивностей потока нейтронов ($s$-процесс), так и в момент взрыва звезды
при высокой нитенсивности потока нейтронов ($r$-процесс).
Однако есть ядра, наблюдаемые в веществе звезд,
которые не могли сформироваться в процессе последовательного присоединения
нейтронов, за что и получили название "обойденные".
Главное препятствие для
образования $\beta$-стабильного обойденного ядра $(A,Z+2)$ в цепочке
последовательных $\beta^-$-превращений представляет энергетический порог
высотой в $(1\div 3) Мэв$ для $\beta$-перехода
$(A,Z)\stackrel{\beta^-}{\longrightarrow} (A,Z+1),$
поскольку праматеринское ядро $(A,Z)$ также $\beta$-стабильно (см. рис.\ref{SH}).

\begin{figure}
\vspace{18 true cm}
\caption{{
Схема цепочки \be-превращений с участием столкновительного \be-распада,
приводящая к образованию обойденного ядра $(A,Z+2)$.}}
\label{SH}
\end{figure}


Попытки решить указанную проблему, не опираясь на стандартную теорию,
предпринимались в работах
%[6-9].
\cite{burbidge}-\cite{frank}.
В них анализировалась возможность получения обойденных изотопов за счет захвата протонов соседними ядрами $(A-1,Z+1)$.
Проведенные оценки показали, что в результате таких процессов как с тепловыми, так и нетепловыми протонами выход
обойденных ядер будет крайне мал \cite{burbidge}-\cite{frank}. Это не позволяет считать $p$-процесс главенствующим, хотя
он, в принципе, и не исключен, играя свою, пусть и малую роль.

В настоящее время считается \cite{97}, что образование обойденных ядер невозможно только в каком-то одном процессе, и их
синтез, по-видимому, происходит только на последней, катастрофической стадии эволюции массивных звезд за счет поглощения
высокоэнергетичных фотонов в реакции $(\gamma, n)$ с некоторыми модификациями за счет реакций $(\gamma, p)$ и $(\gamma,
\alpha)$, либо под действием потока нейтринного излучения от коллапсирующего ядра звезды в реакции $(\nu, e^-)$.
Последний оригинальный физический механизм преoдоления энергетического порога был предложен в \cite{nadeghin}.
% В нем требуемый $\beta^-$-переход осуществляется при облучении соседних
% изобар интенсивным нейтринным потоком.
Авторам  \cite{nadeghin} удалось неплохо воспроизвести ход экспериментальной кривой относительных распространенностей
обойденных ядер. Однако, использование сильно огрубленных оценок величин ядерных матричных элементов для коллективных
$\beta$-переходов (в изотопаналоговые или гамов-теллеровские резонансные состояния дочерних ядер) и, главное,
необходимость рассматривать катастрофическую стадию эволюции звезды (гравитационный коллапс), чтобы обеспечить требуемые
параметры нейтринного потока, не позволяют считать проблему закрытой. Более того, вообще неясно можно ли получить не
относительные, а абсолютные значения распространенностей, отталкиваясь от праматеринских ядер $(A,Z)$, используя
указанный механизм. Несмотря на это, в настоящее время такой путь синтеза обойденных ядер на этапе взрыва сверхновых
звезд, за отсутствием альтернативы считается наиболее вероятным \cite{kosmos, 97}.




Процесс СБР стабильных ядер, о котором говорилось выше, для нуклидов главной
последовательности
предоставляет еще одну возможность преодолеть энергетический порог
и осуществить переход
$(A,Z)\stackrel{\beta^-}{\longrightarrow} (A,Z+1)$, открывая путь к
последующему естественному $\beta$-переходу
$(A,Z+1)\stackrel{\beta^-}{\longrightarrow} (A,Z+2)$.
Расчеты показывают, что модель синтеза обойденных элементов в звездном
веществе на этапе квазиравновесной стадии, основанная на явлении СБР
стабильных ядер главной последовательности, качественно, а в ряде случаев и
количественно, способна воспроизвести нерегулярный ход кривой относительной
распространенности обойденных ядер. Этот факт можно расценивать как косвенное
свидетельство в пользу реальности явления столкновительного $\beta$-распада
стабильных ядер. Однако оценка абсолютных распространенностей обойденных ядер,
образованных в этой модели, показывает, что
результат существенно зависит от временной протяженности квазиравновесных
этапов звездной эволюции при необходимых температурных параметрах,
а эти данные довольно неопределенны.

Поэтому для решения проблемы происхождения обойденных изотопов мы исследуем
также
альтернативный механизм, известный ранее, но не рассматриваемый по отношению к
этой проблеме: $\beta$-распад стабильных ядер, инициируемый
поглощением теплового электромагнитного излучения.
Как показали расчеты, этот процесс может играть существенную роль на
квазиравновесной
стадии звездной эволюции и протекает скорее всего в красных гигантах
и сверхгигантах после образования там ядер $s$-процесса при $T>10^9 K$.
Полученное хорошее согласие
величин не только относительных, но уже и абсолютных распространенностей
обойденных изотопов
с экспериментальными дает основание рассматривать процесс
$\beta$-распада, инициируемый
поглощением теплового электромагнитного излучения
как
лидирующий среди механизмов образования обойденных ядер,
которые также могут вносить свой вклад.

Также рассматривается еще один возможный канал в процессе
синтеза обойденных ядер в звездном веществе,
в котором  $\beta$-распад праматеринского ядра
$(A,Z)$ происходит из термически возбужденных состояний в условиях
статистического равновесия. Этот физический механизм
 похож, но не тождественен процессам, индуцированным столкновениями и
 электромагнитным излучением, так как подразумевает двухступенчатость
процесса, когда на первом этапе материнское ядро переводится в возбужденное
состояние с энергией выше $\beta$-распадного порога, а на втором этапе
происходит $\beta$-переход. Высокие температуры звездной среды
($T \sim 5\cdot 10^9 K$),
достигаемые в процессе звездной эволюции, и наличие у
материнского ядра возбужденных состояний с подходящими квантовыми
характеристиками говорят о возможности такого механизма, но его
эффективность сильно зависит от временной протяженности, а она, по-видимому,
невелика.
Тем не менее возможный вклад процесса $\beta$-распада из возбужденного
состояния праматеринского ядра не исключается.


%В \textbf{разделе 3.5}  обсуждается роль катастрофической стадии звездной
%эволюции в процессе нуклеосинтеза обойденных изотопов. Напомним, что ранее
%только она и рассматривалась как возможная. Показывается, что в условиях
%взрыва сверхновых звезд при достаточно высоких плотностях потока нейтронов
%$(10^{27}--10^{40} нейтрон\cdot см^{-2} с^{-1})$ вероятен столкновительный
%$\beta$-распад, стимулированный нейтрон-ядерными столкновениями.


Цель диссертационной работы -- теоретически исследовать  новый физический
процесс -- столкновительный $\beta$-распада стабильных
ядер и разработать физические модели образования обойденных ядер в звездном
веществе, основанные на $\beta$-распадном механизме ядер в условиях
квазиравновесного этапа эволюции звезд.


В I главе проведено теоретическое исследование процесса столкновительного
$\beta$-распада ядер при их кулоновских столкновениях.
Получено выражение для дифференциального сечения процесса с
точными нерелятивистскими кулоновскими волновыми
функциями для относительного движения в диядерной
системе (случай непрерывного спектра).
Исследована
зависимость полного сечения процесса как функции
относительной энергии столкновения и пороговой энергии
$\Delta$. Проведено сравнение результатов данного расчета и
полученных в борновском приближении.


Во II главе исследован процесс столкновительного $\beta$-распада
         стабильных ядер,
         стимулированный нейтрон-ядерными столкновениями.
Получено выражение для дифференциального сечения процесса $\beta$-распада
стабильного
ядра, инициированного его столкновениями с нуклонами. Волновая функция
относительного движения
в нуклон-ядерной системе находится из уравнения Шредингера с оптическим
потенциалом. Амплитуда
бета-перехода рассчитывается в координатном представлении с использованием
мультипольных разложений
по относительной координате  оператора перехода и волновых функций
непрерывного спектра.
Получено замкнутое выражение для полного сечения процесса как функции
пороговой
энергии и энергии столкновения. Исследовано поведение этого сечения для
нейтрон-ядерных столкновений в широком
диапазоне энергий относительного движения.


В III главе в рамках стандартной теории происхождения элементов
         предлагается новый подход к решению известной
         астрофизической проблемы образования обойденных ядер в
         процессе эволюции массивных звезд. Основу модели
         составляет ранее неизвестное явление -- $\beta$-распад
         стабильного ядра, стимулированный столкновительными
         процессами в звездном веществе. Основное внимание
         уделяется квазиравновесному этапу термоядерной эволюции
         массивных звезд. За счет столкновительного
         $\beta$-распада стабильных ядер из главной
         последовательности преодолеваются энергетические
         барьеры, прерывающие на этих ядрах цепочки
         последовательных $\beta$-превращений и препятствующие
         образованию обойденных ядер. Оценивается роль двух
         возможных каналов преодоления этих барьеров: за счет
         кулоновских столкновений праматеринских ядер с другими
         ядрами среды и за счет столкновений с нейтронами.
         Рассчитываются относительные
         распространенности обойденных изотопов. Оценивается роль
         предлагаемого физического механизма на стадии разогрева
         звезды.
         Рассматриваются также альтернативные механизмы преодоления
         энергетического барьера -- $\beta$-распад из возбужденного состояния
         ядра и $\beta$-распад, инициируемый электромагнитным излучением.
         Рассчитываются относительные и абсолютные значения распространенностей
         (по отношению к распространенностям соответствующих праматеринских
         нуклидов в цепочке $\beta$-распадов).
%                  По интенсивности процесса синтеза обойденных
%         ядер проводится хронометрия конечных этапов эволюции
%         массивных звезд.

В Заключении сформулированы основные результаты, полученные в диссертации.

Библиографический список составлен в порядке очередности следования ссылок
в тексте работы.

Настоящая дисертация написана на основе работ \cite{K1, K2, K3, K4, K5, K6,
K7, K8, K9, K10, K11}.



%EOF
