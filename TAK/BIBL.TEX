\pagebreak

\begin{thebibliography}{200}
\addcontentsline{toc}{chapter}{Литература}


\bibitem{braun}
Braun-Munzinger~P., Stachel~J. // Ann. Rev. Nucl. Part. Sci.--1987.--V.37.--P.97.

%\bibitem{баткинэчая}
\bibitem{batkinech}
Баткин~И.С., Копытин~И.В., Пенионжкевич~Ю.Э. // ЭЧАЯ.--1991.--Т.22.--С.512.


%\bibitem{минин}
\bibitem{minin}
Копытин~И.В., Долгополов~М.А., Корнев~А.С., Минин~Л.А.
// ЯФ.--1996.--Т.59.--С.1195.


%\bibitem{хускивадзе}
\bibitem{huskivadze}
Копытин~И.В., Хускивадзе А.А.
// Изв. РАН. Сер. физ.--1997.--Т.61, N 1.--С.54.

%\bibitem{баткин}
\bibitem{batkin}
Баткин И. С., Копытин И. В., Тютина О.В. // ЯФ.--1991.--Т.53.--С.1576.

%\bibitem{каманин}
\bibitem{kamanin}
Каманин~В.В., Куглер~А., Пенионжкевич~Ю.Э. и др. // ЭЧАЯ.--1989.--Т.20.--С.741.

%\bibitem{ЭЧАЯ1989}
%Каманин~В.В., Куглер~А., Пенионжкевич~Ю.Э. и др. // ЭЧАЯ.--1989.--Т.20.--С.741.


%\bibitem{карпов}
\bibitem{karpov}
Копытин~И.В., Долгополов~М.А., Карпов~Э.Г., Чуракова~Т.А.
// ЯФ.--1997.--Т.60.--С.592.

\bibitem{burbidge}
Burbidge E.M., Burbidge G.R., Fowler W.A., Hoyle F. // Rev.Mod.Phys.--1957.--
V.29.--P.547.

%\bibitem{франк}
\bibitem{frank}
Франк-Каменецкий Д.А. // Астрон.журн.--1961.--Т.38.--С.91.

%\bibitem{надежин}
\bibitem{nadeghin}
Домогацкий Г.В., Надежин Д.К. // Астрон. журн.--1978.--Т.55.--С.516.


%\bibitem{космос}
\bibitem{kosmos}
Физика космоса.--М.: Сов. энциклопедия, 1986.--783с.

\bibitem{97}
Wallerstein G., Iben I., Parker P. e.a. Synthesis of the elements in
stars: forty years of progress// Rev. Mod. Phys.-- 1997.--V.69,N4.--P.995-1084.

%\bibitem{айзенберг}
\bibitem{aiz}
Айзенберг И., Грайнер В. Механизмы возбуждения ядра. --М.: Атомиздат. 1973.--348с.

%\bibitem{ландау}
\bibitem{landau}
Ландау~Л.Д., Лифшиц~Е.М. Квантовая механика. М.: Наука, 1989.--768c.

\bibitem{nord}
Nordsieck A. // Phys. Rev.--1954.--V.93.--P.785.

\bibitem{bess}
Bess L.// Phys.Rev.--1950.--V.77.--P.550.


%\bibitem{зоммерфельд}
\bibitem{zomm}
Зоммерфельд А. Строение атома и спектры.--М.: Физматгиз, 1956.


%\bibitem{ахиезер}
\bibitem{ahi}
Ахиезер А.И. Берестецкий В.В. Квантовая электродинамика. --М.: Наука, 1981.--432c.


%\bibitem{БЛП}
\bibitem{BLP}
Берестецкий В.Б., Лифшиц Е.М., Питаевский Л.П. Квантовая электродинамика.--
М.: Наука, 1989.--723с.

\bibitem{K5}
\textit{Копытин И.В., Крыловецкая Т.А.} Модель процесса синтеза обойденных
ядер в звездном веществе. // Междунар. Совещание ``Свойства ядер,
удалённых от долины стабильности'' (XLVII Совещание по ядерной
спектроскопии и структуре атомного ядра), Обнинск, 10-13 июня 1997 г.
Тезисы докладов,--С-Пб., 1997.--C.292.

\bibitem{K6}
\textit{Копытин И.В., Крыловецкая Т.А.} Модель процесса синтеза обойденных
ядер в звездном веществе // Изв. РАН. Сер. физ.--1998.--Т.62, \No 1.--С. 56-
61.

\bibitem{K7}
\textit{Копытин И.В., Крыловецкая Т.А.} Столкновительный  $\beta$-распад ядер в
кулоновском поле и проблема происхождения обойденных изотопов//
ЯФ.--1998.--Т.61., вып. 3(9).--С. 1589-1599.



\bibitem{NPh1986}
Bauer~W., Cassing~W., Mosel~U. e.a. // Nucl. Phys.---1986.---V.A456.---P.159.

\bibitem{NPh1987}
Kaps~R., Cassing~W., Mosel~U., Tohyama~M.Z. // Nucl. Phys. A.---1987.---V.326.---P.97.


\bibitem{IAF1988}
Баткин~И.С., Копытин~И.В., Беркман~М.И. // ЯФ.--1988.--Т.47.--С.1602.


%тоже что каманин
%\bibitem{ЭЧАЯ1989}
%Каманин~В.В., Куглер~А., Пенионжкевич~Ю.Э. и др. // ЭЧАЯ.--1989.--Т.20.--С.741.

\bibitem{IAF1990}
Баткин~И.С., Копытин~И.В., Чернышев~Д.А. // ЯФ.--1990.--Т.51.--С.1028.

\bibitem{ECHA1991}
Баткин~И.С., Копытин~И.В., Пенионжкевич~Ю.Э. // ЭЧАЯ.---1991.---Т.22.---С.512.


%\bibitem{Ходгсон}
\bibitem{Hot}
Ходгсон~П.Е. Оптическая модель упругого рассеяния. Пер. с англ.---М.: Атомиздат, 1966.

%\bibitem{Балашов}
\bibitem{Bal}
Балашов~В.В. Квантовая теория столкновений. -- М.: Изд-во Моск. ун-та, 1985.--199c.


\bibitem{kor7}
Копытин~И.В., Долгополов~М.А., Корнев~А.С., Чуракова~Т.А.
Электромагнитное излучение при нуклон-ядерном столкновении
// ЯФ.--1997.--Т.60.--С.869.

\bibitem{kor8}
Копытин~И.В., Корнев~А.С., Чуракова~Т.А.
 Эмиссия жестких фотонов при свободно-свободных переходах в диядерной системе
// Изв. АН. Сер. физ.--1997.--Т.61.--С.649.

\bibitem{kor9}
Копытин~И.В., Корнев~А.С.
 Электромагнитное излучение в ядро-ядерных столкновениях
// ЯФ.--1998.--Т.61.--Вып.3.--C.472.

\bibitem{kor10}
Копытин~И.В., Корнев~А.С.
 Дилептонное рождение и эмиссия быстрых позитронов при ядро-ядерных столкновениях
// ЯФ.--1998.--Т.61.--Вып.4.--C.650.


%\bibitem{собельман}
\bibitem{sobelman}
Собельман И.И. Введение в теорию атомных спектров. М.: Наука, 1977.

%\bibitem{давыдов}
\bibitem{davydov}
Давыдов А.С. Теория атомного ядра. М.: Физматгиз, 1958.--611с.


\bibitem{PhR1969}
Becchetti~F.D., Greenlees~G.W. // Phys. Rev.--1969.--V.182.--P.1190.

\bibitem{PhR1971}
Menet~J.J. e.a. // Phys. Rev. C.--1971.--V.4.--P.1114.

\bibitem{PhR1981}
Nadasen~A. e.a. // Phys. Rev. C.--1981.--V.23.--P.1023.

\bibitem{NPh1964}
Wilmore~D., Hodgson~P.E. // Nucl. Phys.--1964.--V.55.--P.673.

\bibitem{PhR1987}
Johnson~C.H. e.a. // Phys. Rev.--1987.--V.C36.--P.2252.

\bibitem{perey}
Perey C.M., Perey F.G. // Atom. data and Nucl. data tables.--1976.--V.17.--P.1.

\bibitem{bechetti}
Becheti F.D., Greenlees G.W. // Phys. Rev.--1969.--V.182.--P.1190.

\bibitem{PhR1970}
Vincent~C.M., Fortune~H.T. // Phys. Rev. C.--1970.--V.2.--P.782.

%\bibitem{абрамовиц}
\bibitem{abramc}
Справочник по специальным функциям./ Под ред. М. Абрамовица и И. Стиган.//
М.: Наука, 1979.--830c.

\bibitem{K8}
\textit{Копытин И.В., Крыловецкая Т.А., Чуракова Т.А.} Столкновительный
бета-распад
 стабильных ядер, стимулированный нуклонами, и его роль в
ядерной астрофизике. // Международное совещание по физике атомного
ядра (XLVIII Совещание по ядерной спектроскопии и структуре атомного
ядра), Москва, 16-19 июня 1998 г. Тезисы докладов,--С-Пб., 1998.--
С.258.

\bibitem{K10}
\textit{Kopytin I.V., Krylovetskaya T.A., Churakova T.A.} Collision-induced
beta-decay of stable nuclei. // Abstracts of contributed papers INPC/98, Paris,
France, 24-28 August 1998.--P.721.

\bibitem{K11}
\textit{Копытин И.В., Крыловецкая Т.А., Чуракова Т.А.} Столкновительный
$\beta$-распад стабильных ядер, стимулированный нейтронами // Изв. РАН.
Сер. физ.--1999.--Т.63, \No 1.--С. 34-38.

%\bibitem{зель_нов}
\bibitem{z_nov}
Зельдович Я.Б., Новиков И.Д. Строение и эволюция Вселенной.// М.: Наука, 1975.--736с.


%\bibitem{зел_бл_ш}
\bibitem{z_b_sh}
Зельдович Я.Б., Блинников С.И., Шакура Н.И. Физические основы строения и эволюции звезд.//
М.: Изд-во Моск. ун-та, 1981.--160с.


%\bibitem{бор}
\bibitem{bor}
Бор О., Моттельсон Б. Структура атомного ядра.// М.: Мир, 1971.



%\bibitem{ленг}
\bibitem{leng}
Ленг К. Астрофизические формулы.// М.: Мир, 1978.



\bibitem{sch12}
Blake J.B., Schramm D.N.// Astrophys. J.--1976.--V.209.--P.846.

\bibitem{sch24}
Colgate S.A. // Astrophys. J.--1971.--V.163.--P.221.

\bibitem{sch39}
Hoyle F., Clayton D.D.// Astrophys. J.--1974.--V.191.--P.705.

\bibitem{sch49}
LeBlanc J.M., Wilson J.R.// Astrophys. J.--1970.--V.161.--P.541.

\bibitem{sch56}
Meier D.L., Epstein R.I., Arnett W.D., Schramm D.N.//  Astrophys. J.--1976.--V.204.--P.869.

\bibitem{sch69}
Schramm D.N., Barkat Z.//  Astrophys. J.--1972.--V.173.--P.195.

\bibitem{sch48}
Lattimer J., Schramm D.N.// Astrophys. J. Lett.--1974.--V.192.--P.L145.

\bibitem{sch34}
Hillebrandt W., Thielman F.K.// Astron. Astrophys.--1977.--V.58.--P.357.

\bibitem{sch52}
Lee T., Schramm D.N., Wefel J.P., Blake J.B.//  Astrophys. J.--1979.--V.232.--P.854.

\bibitem{sch55}
Maripu S.// At. Data Nucl. Data Tables.--1976.--V.17.--P.489.

\bibitem{sch86}
Truran J.W., Cowan J.J., Cameron A.G.W.// Astrophys. J. Lett.--1978.--V.222.--P.L63.

%\bibitem{джелепов}
\bibitem{dzhel}
Джелепов Б.С., Зырянова Л.Н., Суслов Ю.П. Бета-процессы.//
Ленинград: Наука, 1972.--374с.


%\bibitem{чечев}
\bibitem{chechev}
Чечев В.П., Крамаровский Я.М. // УФН.--1981.--Т.134,ВЫП.3.--С.431-469.

\bibitem{schramm73}
Schramm D.N., Symbalisty E.M. // Rep. Prog. Phys.--1981.--V.44.--P.293.

%\bibitem{ядерная}
\bibitem{iader}
Ядерная астрофизика./ Под. ред. Ч. Барнса, Д. Клейтона, Д.М. Шрамма. М.: Мир, 1986.--520c.

\bibitem{isotopes}
Lederer C.M., Hollander J.M., Perlman J. Table of Isotopes// New York: Wiley, 1967.

\bibitem{LL_stat}
Ландау Л.Д., Лифшиц Е.М. Статистическая физика. // М.: Наука, 1964.--568c.

%\bibitem{кубо}
\bibitem{kubo}
Кубо Р. Статистическая механика. // M.: Мир, 1967.--452c.

%\bibitem{гр_рыжик}
\bibitem{gr_r}
Градштейн И.С., Рыжик И.М. Таблицы интегралов, сумм, рядов и произведений.
М.: Физматгиз,1962.--1100с.


\bibitem{becker}
Becker S.A., Iben I.// Astrophys. J..--1979.--V.232.--P.831.

\bibitem{shaw}
Shaw P.B., Clayton D.D., Michel F.C. //Phys. Rev.--1965.--V.140.--P.1433.


%\bibitem{аллер}
\bibitem{aller}
Аллер Л. Астрофизика. // M.: Иностранная литература, 1955.

%\bibitem{мензел}
\bibitem{menzel}
Основные формулы физики. Под ред. Д. Мензела.// M.: Иностранная литература, 1957.--658c.


\bibitem{l86}
Cameron A.G.W. // Astrophys. J. -- 1959. -- V.130.--P.452.


%\bibitem{соловьев}
\bibitem{solov}
Соловьев В.Г. Теория сложных ядер.// М.: Наука, 1971.--560с.

%\bibitem{рел_астр}
\bibitem{rel_a}
Зельдович Я.Б., Новиков И.Д. Релятивистская астрофизика.// М.: Физматгиз, 1967.--656с.


\bibitem{K1}
\textit{Копытин И.В., Лозовая Т.А.} Проблема происхождения обойденных ядер
и столкновительный бета-распад // Ядерная спектроскопия и структура
атомного ядра. Международное совещание, Дубна, 20-23 апреля
1993 г. Тезисы докладов,--С-Пб., 1993.--С.179.

\bibitem{K2}
\textit{Копытин И.В., Долгополов М.А., Крыловецкая Т.А.} Новый механизм
нуклеосинтеза обойденных ядер // Ядерная спектроскопия и структура
атомного ядра Международное совещание, С.-Петербург, 27-30
июня 1995 г. Тезисы докладов,--С-Пб., 1995.--С.180.

\bibitem{K3}
\textit{Копытин И.В., Долгополов М.А., Крыловецкая Т.А.} Новый механизм
нуклеосинтеза обойденных ядер// Изв. РАН. Сер. физ.--1996.--Т.60, \No 1.--
С. 186-191.

\bibitem{K4}
\textit{Копытин И.В., Крыловецкая Т.А., Ткаченко И.И.} Столкновительная модель
нуклеосинтеза обойденных изотопов: роль ядерных резонансов. //
Международное совещание по физике ядра (XLVI Совещание по ядерной
спектроскопии и структуре атомного ядра), Москва, 19-22 июня
1996 г. Тезисы докладов,--С-Пб., 1996.--С.317.


\bibitem{K9}
\textit{Kopytin I.V., Krylovetskaya T.A.} Collisional beta-decay of stable nuclei and
problem of by-passed isotope synthesis. // Abstracts of contributed papers
INPC/98, Paris, France, 24-28 August 1998.--P.679.







\end{thebibliography}
